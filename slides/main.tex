% ==============================================================================
% Apresentação de TCC - Avaliação Quantitativa do Padrão Circuit Breaker
% Autor: Humberto Laff
% Instituição: Centro de Informática (CIn) - UFPE
% Tema: CIn-UFPE (Vermelho institucional)
% ==============================================================================

\documentclass[aspectratio=169,12pt]{beamer}

% ==============================================================================
% TEMA CIn-UFPE
% ==============================================================================

% Tema base limpo
\usetheme{default}
\useinnertheme{rounded}
\useoutertheme{miniframes}

% Cores institucionais CIn-UFPE
\definecolor{cinred}{RGB}{206, 49, 53}        % Vermelho CIn
\definecolor{cinredark}{RGB}{166, 39, 43}     % Vermelho escuro
\definecolor{cinredlight}{RGB}{236, 89, 93}   % Vermelho claro
\definecolor{ufpegray}{RGB}{64, 64, 64}       % Cinza escuro UFPE
\definecolor{ufpelightgray}{RGB}{245, 245, 245} % Cinza claro

% Cores auxiliares para gráficos
\definecolor{successgreen}{RGB}{46, 139, 87}
\definecolor{failred}{RGB}{180, 40, 40}
\definecolor{fallbackorange}{RGB}{230, 140, 30}
\definecolor{neutralblue}{RGB}{70, 130, 180}

% Aplicar cores ao tema
\setbeamercolor{palette primary}{bg=cinred,fg=white}
\setbeamercolor{palette secondary}{bg=cinredark,fg=white}
\setbeamercolor{palette tertiary}{bg=ufpegray,fg=white}
\setbeamercolor{palette quaternary}{bg=cinred,fg=white}

\setbeamercolor{structure}{fg=cinred}
\setbeamercolor{frametitle}{bg=cinred,fg=white}
\setbeamercolor{title}{fg=cinred}
\setbeamercolor{subtitle}{fg=ufpegray}
\setbeamercolor{author}{fg=ufpegray}
\setbeamercolor{institute}{fg=ufpegray}
\setbeamercolor{date}{fg=ufpegray}

\setbeamercolor{block title}{bg=cinred,fg=white}
\setbeamercolor{block body}{bg=ufpelightgray,fg=black}

\setbeamercolor{block title alerted}{bg=failred,fg=white}
\setbeamercolor{block body alerted}{bg=failred!10,fg=black}

\setbeamercolor{block title example}{bg=successgreen,fg=white}
\setbeamercolor{block body example}{bg=successgreen!10,fg=black}

\setbeamercolor{item}{fg=cinred}
\setbeamercolor{subitem}{fg=cinredark}
\setbeamercolor{itemize item}{fg=cinred}
\setbeamercolor{enumerate item}{fg=cinred}

% Fontes
\setbeamerfont{title}{size=\Large,series=\bfseries}
\setbeamerfont{subtitle}{size=\normalsize}
\setbeamerfont{frametitle}{size=\large,series=\bfseries}
\setbeamerfont{block title}{size=\normalsize,series=\bfseries}

% ==============================================================================
% CONFIGURAÇÕES DE LAYOUT
% ==============================================================================

% Remove navegação padrão
\setbeamertemplate{navigation symbols}{}

% Margens do Beamer - reduzir margens laterais e expandir área de texto
\setbeamersize{text margin left=0.5cm, text margin right=0.5cm}

% Rodapé mais compacto
\setbeamertemplate{footline}{
    \leavevmode%
    \hbox{%
        \begin{beamercolorbox}[wd=.33\paperwidth,ht=1.8ex,dp=0.3ex,center]{palette primary}%
            \usebeamerfont{author in head/foot}\insertshortauthor
        \end{beamercolorbox}%
        \begin{beamercolorbox}[wd=.44\paperwidth,ht=1.8ex,dp=0.3ex,center]{palette secondary}%
            \usebeamerfont{title in head/foot}\insertshorttitle
        \end{beamercolorbox}%
        \begin{beamercolorbox}[wd=.23\paperwidth,ht=1.8ex,dp=0.3ex,right]{palette tertiary}%
            \usebeamerfont{date in head/foot}\insertframenumber{} / \inserttotalframenumber\hspace*{2ex}
        \end{beamercolorbox}%
    }%
    \vskip0pt%
}

% Título do frame mais compacto
\setbeamertemplate{frametitle}{
    \vspace{0.1em}
    \insertframetitle
    \vspace{-0.5em}
    \textcolor{cinredlight}{\rule{\textwidth}{0.8pt}}
    \vspace{-0.3em}
}

% ==============================================================================
% PACOTES
% ==============================================================================

\usepackage[utf8]{inputenc}
\usepackage[T1]{fontenc}
\usepackage[brazil]{babel}
\usepackage{graphicx}
\usepackage{booktabs}
\usepackage{amsmath}
\usepackage{tikz}
\usepackage{xcolor}
\usepackage{colortbl}
\usepackage{array}

% Configuração de imagens
\graphicspath{{../tcc_latex/images/}{./}}

% ==============================================================================
% INFORMAÇÕES DO DOCUMENTO
% ==============================================================================

\title[Circuit Breaker em Microsserviços]{Avaliação Quantitativa do Padrão\\Circuit Breaker em Microsserviços}
\subtitle{Um Estudo Empírico sobre Resiliência e Performance}
\author[Humberto Laff]{Humberto Laff}
\institute[CIn-UFPE]{
    Centro de Informática (CIn)\\
    Universidade Federal de Pernambuco (UFPE)
}
\date{Dezembro de 2024}

% Logo no título
\titlegraphic{
    \includegraphics[height=1.2cm]{logo_cin_ufpe.png}
}

% ==============================================================================
% INÍCIO DO DOCUMENTO
% ==============================================================================

\begin{document}

% ==============================================================================
% SLIDE DE TÍTULO
% ==============================================================================
{
\setbeamertemplate{footline}{} % Remove rodapé do título
\begin{frame}
    \begin{center}
        \vspace{-1em}
        \includegraphics[height=1.5cm]{logo_cin_ufpe.png}
        
        \vspace{1.5em}
        
        {\usebeamerfont{title}\usebeamercolor[fg]{title}\inserttitle\par}
        
        \vspace{0.5em}
        
        {\usebeamerfont{subtitle}\usebeamercolor[fg]{subtitle}\insertsubtitle\par}
        
        \vspace{1.5em}
        
        {\usebeamerfont{author}\usebeamercolor[fg]{author}\insertauthor\par}
        
        \vspace{0.5em}
        
        {\footnotesize\usebeamercolor[fg]{institute}\insertinstitute\par}
        
        \vspace{1em}
        
        {\footnotesize\usebeamercolor[fg]{date}\insertdate\par}
    \end{center}
\end{frame}
}

% ==============================================================================
% AGENDA
% ==============================================================================
\begin{frame}{Agenda}
    \tableofcontents
\end{frame}

% ==============================================================================
% SEÇÃO 1: CONTEXTUALIZAÇÃO E PROBLEMA
% ==============================================================================
\section{Contextualização e Problema}

\begin{frame}[shrink=30]{Contextualização}
    \begin{columns}[T]
        \begin{column}{0.55\textwidth}
            \textbf{Era dos Microsserviços}
            \begin{itemize}
                \item Arquitetura predominante em sistemas de grande escala
                \item Comunicação síncrona (REST/HTTP) como padrão
                \item Flexibilidade + Escalabilidade + Deploy independente
            \end{itemize}
            
            \vspace{0.2em}
            \textbf{O Preço da Comunicação Síncrona}
            \begin{itemize}
                \item \textcolor{failred}{\textbf{Acoplamento temporal}} entre serviços
                \item Requisições bloqueadas aguardando resposta
                \item Timeout = recursos desperdiçados
            \end{itemize}
        \end{column}
        \begin{column}{0.42\textwidth}
            \begin{alertblock}{Custo do Downtime}
                Em sistemas bancários:\\[0.2em]
                \textbf{US\$ 5.600 - US\$ 9.000/min}
            \end{alertblock}
            
            \vspace{0.2em}
            \includegraphics[width=0.85\textwidth]{arquitetura_simplificada.png}
        \end{column}
    \end{columns}
\end{frame}

\begin{frame}[shrink=5]{O Problema: Falhas em Cascata}
    \begin{center}
        \textbf{\large O que acontece quando uma dependência falha?}
    \end{center}
    
    \vspace{0.3em}
    
    \begin{columns}[T]
        \begin{column}{0.48\textwidth}
            \begin{block}{Cenário de Falha}
                \centering
                \texttt{[Usuário]}\\
                $\downarrow$\\
                \texttt{[servico-pagamento]}\\
                $\downarrow$ \textcolor{failred}{\textbf{BLOQUEADO}}\\
                \texttt{[servico-adquirente]}\\
                $\downarrow$ \textcolor{failred}{\textbf{TIMEOUT}}\\
                \texttt{[Gateway Externo]}
            \end{block}
        \end{column}
        \begin{column}{0.48\textwidth}
            \begin{alertblock}{Consequências}
                \begin{enumerate}
                    \item \textbf{Thread Pool Starvation}\\
                          {\small Threads bloqueadas aguardando}
                    \item \textbf{Efeito Dominó}\\
                          {\small A falha de C derruba B e A}
                    \item \textbf{Sistema Inutilizável}\\
                          {\small Cascata de falhas}
                \end{enumerate}
            \end{alertblock}
        \end{column}
    \end{columns}
    
    \vspace{0.5em}
    \begin{center}
        \colorbox{cinred!10}{\textcolor{cinred}{\textbf{Problema:} Falhas em cascata causadas por comunicação síncrona entre microsserviços}}
    \end{center}
\end{frame}

% ==============================================================================
% SEÇÃO 2: SOLUÇÃO - CIRCUIT BREAKER
% ==============================================================================
\section{Solução: O Padrão Circuit Breaker}

\begin{frame}[shrink=5]{O Padrão Circuit Breaker}
    \begin{columns}[T]
        \begin{column}{0.55\textwidth}
            \textbf{Analogia:} Funciona como um \textbf{disjuntor elétrico}
            \begin{itemize}
                \item Detecta condições anormais
                \item Interrompe o fluxo para proteger o sistema
                \item Permite recuperação automática
            \end{itemize}
            
            \vspace{0.8em}
            \textbf{Máquina de Estados:}
            \begin{enumerate}
                \item \textcolor{successgreen}{\textbf{FECHADO:}} Operação normal
                \item \textcolor{failred}{\textbf{ABERTO:}} Fail-fast + Fallback
                \item \textcolor{fallbackorange}{\textbf{SEMIABERTO:}} Período de teste
            \end{enumerate}
        \end{column}
        \begin{column}{0.42\textwidth}
            \begin{block}{Configuração Utilizada}
                \footnotesize
                \begin{itemize}
                    \item \texttt{failureRateThreshold:} 50\%
                    \item \texttt{slidingWindowSize:} 10 req
                    \item \texttt{waitDurationInOpenState:} 10s
                    \item \texttt{permittedCallsInHalfOpen:} 3
                \end{itemize}
            \end{block}
            
            \vspace{0.5em}
            \begin{exampleblock}{Degradação Graciosa}
                HTTP 500 $\rightarrow$ HTTP 202\\[0.2em]
                {\footnotesize ``Pagamento agendado para processamento posterior''}
            \end{exampleblock}
        \end{column}
    \end{columns}
\end{frame}

% ==============================================================================
% SEÇÃO 3: METODOLOGIA
% ==============================================================================
\section{Metodologia Experimental}

\begin{frame}[shrink=5]{Metodologia: Abordagem Experimental}
    \begin{columns}[T]
        \begin{column}{0.48\textwidth}
            \textbf{Stack Tecnológico}
            \begin{itemize}
                \item Java 17, Spring Boot 3
                \item Resilience4j 2.1.0
                \item Docker Compose
                \item Grafana k6 (load testing)
            \end{itemize}
            
            \vspace{0.8em}
            \textbf{Versões Comparadas}
            \begin{description}
                \item[\textcolor{ufpegray}{\textbf{V1}}] Baseline (sem resiliência)
                \item[\textcolor{cinred}{\textbf{V2}}] Circuit Breaker + Fallback
                \item[\textcolor{neutralblue}{\textbf{V3}}] Retry com Backoff
            \end{description}
        \end{column}
        \begin{column}{0.48\textwidth}
            \begin{block}{5 Cenários de Teste}
                \footnotesize
                \begin{tabular}{ll}
                    \textbf{Normal} & 100\% saudável \\
                    \textbf{Catástrofe} & 100\% falha por 5min \\
                    \textbf{Degradação} & 5\% $\rightarrow$ 50\% falhas \\
                    \textbf{Rajadas} & 3 ondas de falha \\
                    \textbf{Indisponibilidade} & 75\% offline \\
                \end{tabular}
            \end{block}
            
            \vspace{0.5em}
            \begin{exampleblock}{Volume de Dados}
                \centering
                \textbf{+380.000 requisições}\\
                analisadas estatisticamente
            \end{exampleblock}
        \end{column}
    \end{columns}
\end{frame}

% ==============================================================================
% SEÇÃO 4: RESULTADOS
% ==============================================================================
\section{Resultados}

\begin{frame}[shrink=20]{Resultados Consolidados}
    \begin{center}
        \textbf{\large Comparação de Disponibilidade Percebida: V1 vs V2}
    \end{center}
    
    \vspace{0.2em}
    
    \begin{table}
        \centering
        \small
        \begin{tabular}{l >{\centering\arraybackslash}p{2.5cm} >{\centering\arraybackslash}p{2.5cm} >{\centering\arraybackslash}p{2cm}}
            \toprule
            \textbf{Cenário} & \textbf{V1 (Baseline)} & \textbf{V2 (CB)} & \textbf{Ganho} \\
            \midrule
            Catástrofe & 35,7\% & \textcolor{successgreen}{\textbf{95,1\%}} & +59,3pp \\
            Degradação & 75,4\% & \textcolor{successgreen}{\textbf{95,4\%}} & +20,0pp \\
            \rowcolor{cinred!10} Indisponibilidade & 10,5\% & \textcolor{successgreen}{\textbf{99,6\%}} & \textbf{+89,1pp} \\
            Rajadas & 63,0\% & \textcolor{successgreen}{\textbf{96,7\%}} & +33,6pp \\
            Normal & 100,0\% & 100,0\% & 0pp \\
            \bottomrule
        \end{tabular}
    \end{table}
    
    \vspace{0.3em}
    
    \begin{alertblock}{Resultado Principal}
        \centering
        No cenário de \textbf{Indisponibilidade Extrema}, o Circuit Breaker transformou um sistema com \textbf{10,5\%} de disponibilidade em um com \textbf{99,6\%} — uma \textbf{melhoria de 9,5x}!
    \end{alertblock}
\end{frame}

\begin{frame}{Visualização dos Resultados}
    \vspace{-0.5em}
    \begin{center}
        \includegraphics[width=0.92\textwidth,height=0.82\textheight,keepaspectratio]{01_success_rates_comparison.png}
    \end{center}
\end{frame}

\begin{frame}[shrink=20]{Impacto do Fallback}
    \begin{columns}[T]
        \begin{column}{0.48\textwidth}
            \textbf{Transformação de Erros}
            \begin{itemize}
                \item V1: HTTP 500 (erro fatal)
                \item V2: HTTP 202 (fallback)
            \end{itemize}
            
            \vspace{0.3em}
            \textbf{Contribuição do Fallback:}
            \begin{itemize}
                \item Catástrofe: 62,1\%
                \item Rajadas: 34,6\%
                \item Indisponibilidade: \textbf{99,1\%}
            \end{itemize}
            
            \vspace{0.3em}
            \begin{exampleblock}{Redução de Downtime}
                Catástrofe: 8,37min $\rightarrow$ 0,64min\\
                \textbf{Redução de 92\%}
            \end{exampleblock}
        \end{column}
        \begin{column}{0.50\textwidth}
            \includegraphics[width=\textwidth]{08_fallback_contribution.png}
        \end{column}
    \end{columns}
\end{frame}

\begin{frame}{Análise Consolidada: Rajadas e Catástrofe}
    \vspace*{-1.5em}
    \begin{center}
        \makebox[\textwidth]{\includegraphics[width=1.08\textwidth,height=0.90\textheight,keepaspectratio]{05_combined_summary.png}}
    \end{center}
\end{frame}

% ==============================================================================
% SEÇÃO 5: ANÁLISE ESTATÍSTICA
% ==============================================================================
\section{Análise Estatística}

\begin{frame}[shrink=20]{Validação Estatística}
    \begin{columns}[T]
        \begin{column}{0.50\textwidth}
            \begin{table}
                \centering
                \small
                \begin{tabular}{lr}
                    \toprule
                    \textbf{Métrica} & \textbf{Valor} \\
                    \midrule
                    Teste t (p-value) & p $<$ 0,0001 \\
                    \rowcolor{cinred!15} \textbf{Cohen's d} & \textbf{1,078} \\
                    \midrule
                    ANOVA (F) & 546,79 \\
                    Eta-quadrado ($\eta^2$) & 0,267 \\
                    \midrule
                    IC 95\% (V1) & [495; 508] ms \\
                    IC 95\% (V2) & [400; 410] ms \\
                    \bottomrule
                \end{tabular}
            \end{table}
            
            \vspace{0.2em}
            \begin{alertblock}{Interpretação}
                Cohen's d = 1,078\\
                = Efeito \textbf{GRANDE}\\
                {\small (limiar: d $>$ 0,8)}
            \end{alertblock}
        \end{column}
        \begin{column}{0.48\textwidth}
            \includegraphics[width=\textwidth]{heatmap_success_rate.png}
        \end{column}
    \end{columns}
\end{frame}

% ==============================================================================
% SEÇÃO 6: ANÁLISE V3 (RETRY)
% ==============================================================================
\section{Análise do Padrão Retry (V3)}

\begin{frame}[shrink=20]{Por que Retry (V3) não é Suficiente?}
    \begin{columns}[T]
        \begin{column}{0.50\textwidth}
            \begin{table}
                \centering
                \small
                \begin{tabular}{lrrr}
                    \toprule
                    \textbf{Cenário} & \textbf{V1} & \textbf{V3} & \textbf{V2} \\
                    \midrule
                    Indisp. & 10,5\% & 15,7\% & \textcolor{successgreen}{\textbf{99,6\%}} \\
                    Catástrofe & 35,7\% & 52,8\% & \textcolor{successgreen}{\textbf{95,1\%}} \\
                    Rajadas & 63,0\% & 77,7\% & \textcolor{successgreen}{\textbf{96,7\%}} \\
                    \bottomrule
                \end{tabular}
            \end{table}
            
            \vspace{0.3em}
            \begin{alertblock}{Problemas do Retry}
                \begin{enumerate}
                    \item Não melhora em falhas persistentes
                    \item Aumenta latência
                    \item Pode amplificar carga (3x)
                \end{enumerate}
            \end{alertblock}
        \end{column}
        \begin{column}{0.48\textwidth}
            \begin{exampleblock}{Recomendação}
                \textbf{Retry} deve ser usado \textbf{dentro} do Circuit Breaker, não como substituto.
                
                \vspace{0.3em}
                {\small CB fechado $\rightarrow$ retry permitido\\
                CB aberto $\rightarrow$ fallback imediato}
            \end{exampleblock}
            
            \vspace{0.3em}
            \includegraphics[width=\textwidth]{error_rates.png}
        \end{column}
    \end{columns}
\end{frame}

% ==============================================================================
% SEÇÃO 7: CONCLUSÕES
% ==============================================================================
\section{Conclusões}

\begin{frame}[shrink=5]{Contribuições do Trabalho}
    \begin{enumerate}
        \item \textbf{Evidência Empírica Robusta}\\
              {\small Dados quantitativos de 5 cenários realistas (+380.000 requisições)}
              
        \vspace{0.5em}
        \item \textbf{Comparação entre Padrões}\\
              {\small CB (V2) vs Retry isolado (V3) — demonstrou superioridade do CB}
              
        \vspace{0.5em}
        \item \textbf{Análise Estatística Rigorosa}\\
              {\small Cohen's d = 1,078 (efeito grande), ANOVA significativa}
              
        \vspace{0.5em}
        \item \textbf{Metodologia Reprodutível}\\
              {\small Framework Docker + k6 disponível para replicação}
              
        \vspace{0.5em}
        \item \textbf{Quantificação de Benefícios}\\
              {\small Até \textbf{+89pp} na disponibilidade, \textbf{-99,5\%} nas falhas}
    \end{enumerate}
\end{frame}

\begin{frame}[shrink=5]{Limitações e Trabalhos Futuros}
    \begin{columns}[T]
        \begin{column}{0.48\textwidth}
            \begin{alertblock}{Limitações}
                \begin{itemize}
                    \item POC simplificada
                    \item Ambiente local
                    \item Configuração fixa do CB
                    \item Carga sintética (k6)
                \end{itemize}
            \end{alertblock}
        \end{column}
        \begin{column}{0.48\textwidth}
            \begin{exampleblock}{Trabalhos Futuros}
                \begin{itemize}
                    \item Combinação CB + Retry
                    \item Ambiente cloud distribuído
                    \item Múltiplas dependências
                    \item CB adaptativos com ML
                    \item Service Meshes (Istio)
                \end{itemize}
            \end{exampleblock}
        \end{column}
    \end{columns}
\end{frame}

\begin{frame}[shrink=5]{Conclusão}
    \begin{center}
        \Large
        \textit{``O Circuit Breaker transformou um sistema\\
        inutilizável (10,5\%) em um altamente disponível (99,6\%).\\[0.3em]
        Isso representa uma \textbf{melhoria de 9,5x}.''}
    \end{center}
    
    \vspace{1.5em}
    
    \begin{columns}[c]
        \begin{column}{0.32\textwidth}
            \centering
            \textcolor{successgreen}{\Large\textbf{CB é essencial}}\\[0.3em]
            {\small para microsserviços\\síncronos críticos}
        \end{column}
        \begin{column}{0.32\textwidth}
            \centering
            \textcolor{failred}{\Large\textbf{Retry sozinho}}\\[0.3em]
            {\small não é suficiente para\\falhas persistentes}
        \end{column}
        \begin{column}{0.32\textwidth}
            \centering
            \textcolor{fallbackorange}{\Large\textbf{Fallback}}\\[0.3em]
            {\small preserva a experiência\\do usuário}
        \end{column}
    \end{columns}
\end{frame}

% ==============================================================================
% SLIDE FINAL
% ==============================================================================
{
\setbeamertemplate{footline}{} % Remove rodapé
\begin{frame}
    \begin{center}
        \vspace{1em}
        
        {\Huge\textcolor{cinred}{\textbf{Obrigado!}}}
        
        \vspace{2em}
        
        {\Large\textcolor{ufpegray}{Perguntas?}}
        
        \vspace{2em}
        
        \includegraphics[height=1.2cm]{logo_cin_ufpe.png}
        
        \vspace{1.5em}
        
        \textbf{Humberto Laff}\\
        {\small hlaff@cin.ufpe.br}\\[0.5em]
        {\footnotesize Centro de Informática (CIn) — UFPE}
    \end{center}
\end{frame}
}

\end{document}
