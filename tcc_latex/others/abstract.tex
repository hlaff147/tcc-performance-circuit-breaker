\begin{resumo}[Abstract]
\begin{otherlanguage*}{english}
Microservices architectures have become the standard for building scalable distributed systems, yet synchronous inter-service communication introduces critical vulnerabilities: when a downstream dependency degrades or fails, cascading failures can propagate throughout the entire system, leading to thread pool exhaustion and complete service unavailability. While the Circuit Breaker pattern is widely recommended as a mitigation strategy, empirical evidence quantifying its effectiveness under realistic failure scenarios remains scarce in the literature. This work addresses this gap by conducting a controlled experimental study comparing a baseline architecture against one protected by Circuit Breaker (implemented with Resilience4j) across four stress scenarios: catastrophic failure, gradual degradation, intermittent bursts, and extreme unavailability. Using Docker-orchestrated microservices and k6 load testing with over 769,000 total requests (considering V1 and V2), we measured throughput, latency (p95/p99), and success rates. Results demonstrate that the Circuit Breaker delivers substantial resilience gains: in the extreme unavailability scenario (75\% downtime), success rates improved from 10.14\% to 97.08\% (+86.94 percentage points), representing a 96.77\% reduction in failures. Additionally, average response time decreased from 735.05ms to 477.60ms (35\% improvement), and median response time dropped from 160.56ms to 35.69ms (78\% improvement). Statistical analysis (Mann-Whitney U test, p $<$ 0.001; Cliff's Delta = 0.38 --- medium effect size) confirms significant difference between versions. This study provides reproducible empirical evidence and a reusable experimental framework for evaluating fault tolerance patterns in synchronous microservices architectures.

\vspace{\onelineskip}
\noindent
\textbf{Keywords:} Microservices. Circuit Breaker. Resilience. Fault Tolerance. Distributed Systems. Performance Engineering. Resilience4j.
\end{otherlanguage*}
\end{resumo}
