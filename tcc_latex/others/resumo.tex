\begin{resumo}
Arquiteturas de microsserviços tornaram-se o padrão para construção de sistemas distribuídos escaláveis, porém a comunicação síncrona entre serviços introduz vulnerabilidades críticas: quando uma dependência downstream degrada ou falha, falhas em cascata podem se propagar por todo o sistema, levando à exaustão do pool de threads e indisponibilidade total do serviço. Embora o padrão Circuit Breaker seja amplamente recomendado como estratégia de mitigação, evidências empíricas quantificando sua eficácia em cenários realistas de falha permanecem escassas na literatura. Este trabalho preenche essa lacuna através de um estudo experimental controlado comparando uma arquitetura baseline contra uma protegida por Circuit Breaker (implementado com Resilience4j) em quatro cenários de estresse: falha catastrófica, degradação gradual, rajadas intermitentes e indisponibilidade extrema. Utilizando microsserviços orquestrados via Docker e testes de carga com k6 totalizando mais de 769.000 requisições, foram medidos vazão, latência (p95/p99) e taxas de sucesso. Os resultados demonstram que o Circuit Breaker proporciona ganhos substanciais de resiliência: no cenário de indisponibilidade extrema (75\% de downtime), as taxas de sucesso melhoraram de 10,14\% para 97,08\% (+86,94 pontos percentuais), representando uma redução de 96,77\% nas falhas. Adicionalmente, o tempo médio de resposta reduziu de 735,05ms para 477,60ms (melhoria de 35\%), e a mediana caiu de 160,56ms para 35,69ms (melhoria de 78\%). A análise estatística (teste de Mann-Whitney U, p $<$ 0,001; Cliff's Delta = 0,38 --- efeito médio) confirma diferença significativa entre as versões. Este estudo fornece evidência empírica reprodutível e um framework experimental reutilizável para avaliação de padrões de tolerância a falhas em arquiteturas de microsserviços síncronos.

\vspace{\onelineskip}
\noindent
\textbf{Palavras-chave:} Microsserviços. Circuit Breaker. Resiliência. Tolerância a Falhas. Sistemas Distribuídos. Engenharia de Desempenho. Resilience4j.
\end{resumo}
