\documentclass[12pt, openright, oneside, a4paper, chapter=TITLE, section=TITLE, english, brazil]{abntex2/abntex2}

\renewcommand{\baselinestretch}{1.5}
\usepackage{abntex2/abntex2-cin-ufpe}
\renewcommand*\arraystretch{1.2}
\usepackage{pdfpages}
\usepackage{float}
\usepackage{cmap}
\usepackage{lmodern}
\usepackage[T1]{fontenc}
\usepackage[utf8]{inputenc}
\usepackage{lastpage}
\usepackage{indentfirst}
\usepackage{graphicx}
\usepackage[versalete,alf,abnt-and-type=e,abnt-etal-list=0,abnt-etal-cite=3]{abntex2/abntex2cite}
\usepackage{multirow}
\usepackage[section]{placeins}
\usepackage[acronym,nonumberlist,nogroupskip,noredefwarn]{glossaries}

% Configurações de Glossário (Estilo CIn)
\newcolumntype{L}[1]{>{\raggedright\let\newline\\\arraybackslash\hspace{0pt}}m{#1}}
\newglossarystyle{modsuper}{%
  \glossarystyle{super}%
  \renewcommand{\glsgroupskip}{}
  \renewenvironment{theglossary}%
  {\begin{longtable}{L{0.2\textwidth}L{0.8\textwidth}}}%
  {\end{longtable}}%
}

% Inputs de Configuração
% Dados do Trabalho
\titulo{Análise de Desempenho e Resiliência em Microsserviços Síncronos: Um Estudo Experimental do Padrão Circuit Breaker}
\autor{Humberto L. A. Fonseca Filho}
\local{Recife}
\data{2025}

% Dados da Instituição
\instituicao{Universidade Federal de Pernambuco}
\departamento{Centro de Informática}
\programa{Graduação em Ciência da Computação} % Ajuste conforme necessário
\emailprograma{secgrad@cin.ufpe.br} % Ajuste conforme necessário

% Dados da Orientação
\orientador{Jamilson Ramalho Dantas}

% Dados do Tipo de Trabalho
\tipotrabalho{Trabalho de Conclusão de Curso}
\preambulo{Trabalho de Conclusão de Curso apresentado ao Centro de Informática da Universidade Federal de Pernambuco como requisito parcial para obtenção do grau de Bacharel em Ciência da Computação.}

\usepackage{url}
\usepackage{booktabs}
\usepackage{listings}
\usepackage{xcolor}
\usepackage{inconsolata}
\usepackage{adjustbox}
\usepackage{float}

% Cores para listings
\definecolor{pblue}{rgb}{0.13,0.13,1}
\definecolor{javagreen}{rgb}{0,0.5,0}
\definecolor{javapurple}{rgb}{0.5,0,0.5}
\definecolor{javadocblue}{rgb}{0,0,1}
\definecolor{cinza}{HTML}{FCF8F8}

% Configuração do listings
\lstset{
    language=Java,
    basicstyle=\ttfamily\scriptsize,
    keywordstyle=\color{javapurple}\bfseries,
    stringstyle=\color{pblue},
    commentstyle=\color{javagreen},
    morecomment=[s][\color{javadocblue}]{/**}{*/},
    morecomment=[s][\color{gray}]{@}{\ },
    numbers=left,
    numberstyle=\tiny\color{black},
    backgroundcolor=\color{cinza},
    stepnumber=1,
    numbersep=8pt,
    xleftmargin=14pt,
    tabsize=4,
    showspaces=false,
    showstringspaces=false,
    breaklines=true,
    frame=single,
    rulecolor=\color{black}
}


\makeindex
\makenoidxglossaries

% Início do Documento
\begin{document}
\frenchspacing
\imprimircapa
\imprimirfolhaderosto*~
% Inserir aqui inputs de: ficha, ata_defesa, dedicatoria, agradecimentos, epigrafe, resumo
\begin{resumo}
Arquiteturas de microsserviços tornaram-se o padrão para construção de sistemas distribuídos escaláveis, porém a comunicação síncrona entre serviços introduz vulnerabilidades críticas: quando uma dependência downstream degrada ou falha, falhas em cascata podem se propagar por todo o sistema, levando à exaustão do pool de threads e indisponibilidade total do serviço. Embora o padrão Circuit Breaker seja amplamente recomendado como estratégia de mitigação, evidências empíricas quantificando sua eficácia em cenários realistas de falha permanecem escassas na literatura. Este trabalho preenche essa lacuna através de um estudo experimental controlado comparando uma arquitetura baseline contra uma protegida por Circuit Breaker (implementado com Resilience4j) em quatro cenários de estresse: falha catastrófica, degradação gradual, rajadas intermitentes e indisponibilidade extrema. Utilizando microsserviços orquestrados via Docker e testes de carga com k6 totalizando mais de 769.000 requisições, foram medidos vazão, latência (p95/p99) e taxas de sucesso. Os resultados demonstram que o Circuit Breaker proporciona ganhos substanciais de resiliência: no cenário de indisponibilidade extrema (75\% de downtime), as taxas de sucesso melhoraram de 10,14\% para 97,08\% (+86,94 pontos percentuais), representando uma redução de 96,77\% nas falhas. Adicionalmente, o tempo médio de resposta reduziu de 735,05ms para 477,60ms (melhoria de 35\%), e a mediana caiu de 160,56ms para 35,69ms (melhoria de 78\%). A análise estatística (teste de Mann-Whitney U, p $<$ 0,001; Cliff's Delta = 0,38 --- efeito médio) confirma diferença significativa entre as versões. Este estudo fornece evidência empírica reprodutível e um framework experimental reutilizável para avaliação de padrões de tolerância a falhas em arquiteturas de microsserviços síncronos.

\vspace{\onelineskip}
\noindent
\textbf{Palavras-chave:} Microsserviços. Circuit Breaker. Resiliência. Tolerância a Falhas. Sistemas Distribuídos. Engenharia de Desempenho. Resilience4j.
\end{resumo}

\begin{resumo}[Abstract]
\begin{otherlanguage*}{english}
Microservices architectures have become the standard for building scalable distributed systems, yet synchronous inter-service communication introduces critical vulnerabilities: when a downstream dependency degrades or fails, cascading failures can propagate throughout the entire system, leading to thread pool exhaustion and complete service unavailability. While the Circuit Breaker pattern is widely recommended as a mitigation strategy, empirical evidence quantifying its effectiveness under realistic failure scenarios remains scarce in the literature. This work addresses this gap by conducting a controlled experimental study comparing a baseline architecture against one protected by Circuit Breaker (implemented with Resilience4j) across four stress scenarios: catastrophic failure, gradual degradation, intermittent bursts, and extreme unavailability. Using Docker-orchestrated microservices and k6 load testing with over 769,000 total requests (considering V1 and V2), we measured throughput, latency (p95/p99), and success rates. Results demonstrate that the Circuit Breaker delivers substantial resilience gains: in the extreme unavailability scenario (75\% downtime), success rates improved from 10.14\% to 97.08\% (+86.94 percentage points), representing a 96.77\% reduction in failures. Additionally, average response time decreased from 735.05ms to 477.60ms (35\% improvement), and median response time dropped from 160.56ms to 35.69ms (78\% improvement). Statistical analysis (Mann-Whitney U test, p $<$ 0.001; Cliff's Delta = 0.38 --- medium effect size) confirms significant difference between versions. This study provides reproducible empirical evidence and a reusable experimental framework for evaluating fault tolerance patterns in synchronous microservices architectures.

\vspace{\onelineskip}
\noindent
\textbf{Keywords:} Microservices. Circuit Breaker. Resilience. Fault Tolerance. Distributed Systems. Performance Engineering. Resilience4j.
\end{otherlanguage*}
\end{resumo}


% Listas
\pdfbookmark[0]{\listfigurename}{lof}
\listoffigures*
\cleardoublepage

\renewcommand{\lstlistlistingname}{Lista de Códigos}
\renewcommand{\lstlistingname}{Código}
\pdfbookmark[0]{\lstlistlistingname}{lol}
\lstlistoflistings*
\cleardoublepage

\pdfbookmark[0]{\listtablename}{lot}
\listoftables*
\cleardoublepage

% \printnoidxglossary[style=modsuper,type=\acronymtype,title={\listadesiglasname},nonumberlist]
% \cleardoublepage

\pdfbookmark[0]{\contentsname}{toc}
\tableofcontents*
\cleardoublepage

\textual
% Capítulos do TCC
\section{Introduction}
\label{sec:introduction}

The microservices architecture has become ubiquitous in organizations building large-scale digital platforms that require continuous availability and accelerated evolution cycles \cite{newman2021}. E-commerce ecosystems and payment processing systems exemplify this movement, demanding flexibility, fault tolerance, and rapid adaptation to variable transaction volumes.

While partitioning functionality into independent services facilitates parallel development and selective scalability, this logical independence relies on real-time interactions between services, typically through REST APIs and declarative clients like Spring Cloud OpenFeign. Synchronous communication simplifies implementation and observability but introduces strong temporal coupling: the consumer service remains blocked until the dependent service responds or a timeout occurs \cite{nygard2018}.

\subsection{Problem Statement}

The risk inherent to synchronous communication constitutes the core of this investigation. When a dependent service experiences high latency or intermittent unavailability, the consuming service waits until timeout, keeping threads blocked. With increasing request volume, the thread pool exhausts (\textbf{thread pool starvation}), causing \textbf{cascading failures} that can bring down the entire system.

This problem is transversal across software engineering domains: e-commerce, logistics, healthcare, fintech, streaming, and IoT. Any distributed system relying on synchronous HTTP calls is subject to the risks analyzed here.

\subsection{Proposed Solution}

The Circuit Breaker (CB) pattern emerges as a response to these challenges. Operating as a state machine with three modes—\textbf{Closed}, \textbf{Open}, and \textbf{Half-Open}—the CB monitors calls to dependent services and interrupts new attempts when failure rates exceed configured thresholds, failing fast and protecting the consumer from resource exhaustion \cite{fowler2014cb}.

\subsection{Contribution}

Despite extensive literature on microservices and resilience patterns, there is a significant gap regarding \textbf{quantitative experimental studies} demonstrating the actual impact of the Circuit Breaker pattern. Most available documentation is limited to conceptual descriptions or trivial examples.

This paper fills this gap by:
\begin{enumerate}
    \item \textbf{Implementing} a Proof of Concept (POC) simulating a microservices ecosystem with synchronous dependency, instrumented with Resilience4j;
    \item \textbf{Executing} benchmark campaigns with Docker and k6 to \textbf{empirically measure} throughput, latency (p95), and error rates across controlled scenarios;
    \item \textbf{Comparing} three architectural versions: Baseline (V1), Circuit Breaker (V2), and Retry with exponential backoff (V3);
    \item \textbf{Providing} rigorous statistical analysis with effect size measures (Cohen's d = 1.078).
\end{enumerate}

\chapter{Fundamentação Teórica}
\label{cap:fundamentacao}

\section{Arquitetura de Microsserviços}
A arquitetura de microsserviços representa uma abordagem de desenvolvimento de software que estrutura uma aplicação como um conjunto de serviços pequenos, autônomos e fracamente acoplados \cite{newman2021}. Cada serviço é responsável por uma capacidade de negócio específica, pode ser desenvolvido, implantado e escalado de forma independente, e se comunica com outros serviços através de interfaces bem definidas, geralmente APIs REST ou mensageria assíncrona \cite{fowler2014}.

Os principais benefícios desta arquitetura incluem: (i) \textbf{escalabilidade seletiva}, permitindo escalar apenas os serviços sob maior demanda; (ii) \textbf{autonomia de equipes}, possibilitando que times diferentes desenvolvam e implantem serviços independentemente; (iii) \textbf{resiliência a falhas}, onde a falha de um serviço não necessariamente compromete todo o sistema; e (iv) \textbf{flexibilidade tecnológica}, permitindo que cada serviço utilize a tecnologia mais adequada ao seu propósito \cite{richardson2018}.

Contudo, essa arquitetura introduz desafios significativos de natureza distribuída. A comunicação entre serviços através da rede é inerentemente não confiável, sujeita a latências variáveis, timeouts e falhas parciais. Em sistemas monolíticos, chamadas de função são executadas em memória com latência desprezível; em microsserviços, cada chamada atravessa a rede e pode falhar de múltiplas formas \cite{nygard2018}.

\subsection{Dependabilidade em Sistemas Distribuídos}
Para compreender a resiliência em microsserviços, é fundamental estabelecer uma base teórica sobre a dependabilidade. Conforme a taxonomia seminal de \citeonline{avizienis2004}, a dependabilidade é definida como a capacidade de evitar falhas de serviço que sejam mais frequentes ou severas do que o aceitável para os usuários. Esse conceito é sustentado por seis atributos fundamentais: \textbf{disponibilidade} (prontidão para serviço correto), \textbf{confiabilidade} (continuidade do serviço correto), \textbf{segurança} (safety), \textbf{confidencialidade}, \textbf{integridade} e \textbf{manutenibilidade}.

O modelo de \textbf{falha-erro-mau funcionamento} (fault-error-failure) fornece a estrutura para analisar a degradação do sistema:
\begin{itemize}
    \item \textbf{Fault (falha):} A causa raiz, que pode ser física, de interação ou de desenvolvimento.
    \item \textbf{Error (erro):} O estado interno anormal resultante dessa falha.
    \item \textbf{Failure (mau funcionamento):} Ocorre quando o serviço deixa de cumprir sua especificação perante o usuário externo.
\end{itemize}

Em microsserviços, as \textbf{falhas de timing} — onde uma resposta correta é entregue fora da janela temporal aceitável — são tão críticas quanto as \textbf{falhas de conteúdo}, especialmente em sistemas de pagamento que operam sob rigorosos Acordos de Nível de Serviço (SLAs) \cite{avizienis2004}.

\section{Comunicação Síncrona e seus Riscos}
A comunicação síncrona, tipicamente implementada via HTTP/REST, é caracterizada pelo bloqueio do serviço consumidor enquanto aguarda a resposta do serviço provedor. Embora seja simples de implementar e depurar, este modelo introduz um \textbf{acoplamento temporal} entre os serviços: se o provedor estiver lento ou indisponível, o consumidor também será afetado \cite{burns2018}.

Os principais riscos associados à comunicação síncrona incluem:

\begin{itemize}
    \item \textbf{Thread Pool Starvation:} Quando múltiplas requisições aguardam respostas lentas, as threads do servidor ficam bloqueadas, esgotando o pool disponível e impedindo o processamento de novas requisições.
    \item \textbf{Falhas em Cascata:} Uma dependência lenta ou indisponível pode propagar sua condição para todos os serviços que dela dependem, amplificando o impacto de uma falha localizada para todo o ecossistema.
    \item \textbf{Efeito Dominó:} Em cadeias de dependências (A → B → C), a falha de C pode derrubar B e, consequentemente, A, mesmo que estes estejam funcionando corretamente.
\end{itemize}

A Tabela \ref{tab:sinc-vs-assinc} compara as características da comunicação síncrona com a assíncrona (Arquitetura Orientada a Eventos), evidenciando os trade-offs de cada abordagem.

\begin{table}[H]
\centering
\caption{Comparação entre comunicação síncrona e assíncrona}
\label{tab:sinc-vs-assinc}
\begin{tabular}{lp{4.5cm}p{4.5cm}}
\toprule
\textbf{Atributo} & \textbf{Comunicação Síncrona} & \textbf{Comunicação Assíncrona (EDA)} \\
\midrule
Acoplamento & Forte (temporal e espacial) & Fraco (baseado em eventos) \\
Latência & Ditada pelo serviço mais lento & Processamento em paralelo \\
Consistência & Imediata (tipicamente) & Eventual \\
Complexidade & Menor (fluxo linear) & Maior (gerenciamento de eventos) \\
Tolerância a Falhas & Exige padrões como Circuit Breaker & Naturalmente resiliente \\
\bottomrule
\end{tabular}
\end{table}

Embora a EDA ofereça vantagens claras em termos de elasticidade e robustez, a comunicação síncrona permanece necessária para operações que exigem confirmação imediata de estado, como a validação de crédito em uma jornada de checkout.

\section{Padrões de Tolerância a Falhas}
Para mitigar os riscos da comunicação distribuída, diversos padrões de tolerância a falhas foram propostos pela comunidade de engenharia de software \cite{nygard2018, richardson2018}. A Tabela \ref{tab:padroes-resiliencia} resume os principais padrões, seus mecanismos e riscos associados.

\begin{table}[H]
\centering
\caption{Comparação de padrões de tolerância a falhas}
\label{tab:padroes-resiliencia}
\begin{tabular}{lp{3.5cm}p{3.5cm}p{3.5cm}}
\toprule
\textbf{Padrão} & \textbf{Mecanismo} & \textbf{Benefício Principal} & \textbf{Risco Associado} \\
\midrule
Timeout & Limite máximo de espera & Liberação de recursos bloqueados & Descarte de requisições válidas \\
Retry & Repetição de chamadas & Recupera falhas transitórias & Amplificação de carga (Retry Storm) \\
Bulkhead & Compartimentalização de recursos & Isolamento de falhas & Overhead de gerenciamento \\
Rate Limiter & Controle de taxa & Proteção contra sobrecarga & Rejeição de requisições legítimas \\
Circuit Breaker & Corte de fluxo baseado em erro & Previne falhas em cascata & Configuração de limiares imprecisa \\
\bottomrule
\end{tabular}
\end{table}

Entre esses padrões, o \textbf{Circuit Breaker} destaca-se por sua capacidade de interromper temporariamente chamadas a uma dependência que apresenta falhas recorrentes, permitindo recuperação e evitando desperdício de recursos. Estudos indicam que a implementação correta deste padrão pode reduzir as taxas de erro em até 58\% durante períodos de instabilidade \cite{ieeecloud2024}.

\section{O Padrão Circuit Breaker em Detalhe}
O padrão Circuit Breaker, popularizado por Michael Nygard em seu livro ``Release It!'' \cite{nygard2018} e documentado por Martin Fowler \cite{fowler2014cb}, opera como um disjuntor elétrico: quando detecta condições anormais, ``abre'' para interromper o fluxo e proteger o sistema.

A implementação típica do Circuit Breaker utiliza uma máquina de estados com três estados:

\begin{enumerate}
    \item \textbf{Fechado (Closed):} Estado normal de operação. Todas as requisições são encaminhadas à dependência. O CB monitora continuamente métricas como taxa de falha e tempo de resposta.
    \item \textbf{Aberto (Open):} Ativado quando métricas excedem limiares configurados (ex: >50\% de falhas). Requisições são imediatamente rejeitadas ou redirecionadas a um fallback, sem tentar contactar a dependência. Após um período de espera (\textit{wait duration}), transiciona para Semiaberto.
    \item \textbf{Semiaberto (Half-Open):} Estado de teste. Um número limitado de requisições é permitido para verificar se a dependência se recuperou. Se bem-sucedidas, retorna ao estado Fechado; caso contrário, volta ao estado Aberto.
\end{enumerate}

\section{Implementações de Circuit Breaker: Hystrix vs Resilience4j}
O Netflix Hystrix foi pioneiro na implementação do padrão Circuit Breaker para a JVM, sendo amplamente adotado na indústria \cite{hystrix, netflix2016}. Contudo, em 2018, o projeto entrou em modo de manutenção, e o Resilience4j emergiu como seu sucessor recomendado \cite{resilience4j}.

O Resilience4j apresenta vantagens significativas sobre o Hystrix:

\begin{itemize}
    \item \textbf{Design modular:} Cada padrão (Circuit Breaker, Retry, Bulkhead, Rate Limiter, Time Limiter) é um módulo independente que pode ser composto conforme necessário.
    \item \textbf{Menor footprint:} Não requer thread pools separados como o Hystrix, reduzindo consumo de recursos.
    \item \textbf{Suporte a programação funcional:} Integração nativa com lambdas Java 8+, CompletableFuture e frameworks reativos.
    \item \textbf{Métricas nativas:} Exportação de métricas para Prometheus/Micrometer sem configuração adicional.
    \item \textbf{Janela deslizante configurável:} Suporta janelas baseadas em contagem ou tempo para cálculo de métricas.
\end{itemize}

\section{Trabalhos Relacionados}
A literatura recente apresenta diversos estudos sobre resiliência em microsserviços. \citeonline{montesi2016} analisam a interação entre Circuit Breakers e API Gateways, propondo padrões de composição. \citeonline{burns2018} contextualiza padrões de resiliência no design de sistemas distribuídos modernos.

De particular relevância para este trabalho é o estudo de \citeonline{pinheiro2024}, que propõe uma modelagem analítica do comportamento de Circuit Breakers utilizando Redes de Petri Estocásticas (SPNs). Esta abordagem permite prever o impacto de diferentes parametrizações do CB em métricas de SLA antes da implantação em produção. O presente TCC complementa essa contribuição teórica ao fornecer \textbf{validação empírica} dos benefícios do Circuit Breaker através de experimentos controlados em um ambiente realista.

A documentação oficial da Microsoft Azure \cite{microsoftpatterns} apresenta o Circuit Breaker como um dos padrões essenciais para aplicações cloud-native, reforçando sua relevância em arquiteturas modernas de larga escala.

\section{Methodology}
\label{sec:methodology}

This work adopts a \textbf{quantitative experimental research} approach. We built a simplified Proof of Concept (POC) simulating a microservices ecosystem with synchronous dependency. The POC is intentionally minimalistic—without database, cache, or authentication—to isolate the Circuit Breaker's effect as the sole variable of interest.

\subsection{Experimental Architecture}

The POC comprises two Spring Boot microservices packaged as Docker containers:

\begin{itemize}
    \item \textbf{payment-service}: Orchestrates the payment flow and synchronously consumes the acquirer service via Feign Client;
    \item \textbf{acquirer-service}: Simulates an external payment gateway with configurable behavior (normal, latency, or failure mode).
\end{itemize}

Figure~\ref{fig:architecture} summarizes the simplified architecture used throughout the experiments.

\begin{figure}[t]
\centering
\includegraphics[width=0.48\textwidth]{simplified_architecture_en.png}
\caption{Simplified architecture of the experimental microservices ecosystem.}
\label{fig:architecture}
\end{figure}

Four versions of payment-service were developed:
\begin{itemize}
    \item \textbf{V1 (Baseline)}: Basic timeouts only (2s);
    \item \textbf{V2 (Circuit Breaker)}: Resilience4j with CB and fallback returning HTTP 202;
    \item \textbf{V3 (Retry)}: Exponential backoff retry (3 attempts, 500ms→1s→2s);
    \item \textbf{V4 (Composition)}: Combines V2 and V3, simulating a production-grade resilience stack.
\end{itemize}

\subsection{Circuit Breaker Configuration (V2)}

The Resilience4j Circuit Breaker was configured to provide a balance between fail-fast responsiveness and stability. Thresholds were selected based on the \textbf{criticality of the payment domain}.

\begin{itemize}
    \item \textbf{failureRateThreshold}: 50\% --- Represents a conservative threshold where the system assumes the dependency is no longer reliable. Lower values (e.g., 20\%) might cause \textit{flapping} due to transient jitters;
    \item \textbf{slowCallRateThreshold}: 70\% --- Prioritizes thread protection against slow dependencies, which are often more dangerous than total failures as they cause silent resource exhaustion;
    \item \textbf{slidingWindowSize}: 20 requests --- Increased from 10 to provide a more robust statistical sample, reducing the impact of outliers;
    \item \textbf{waitDurationInOpenState}: 15s --- Aligned with standard recovery times for cloud-native load balancers to detect healthy instances;
    \item \textbf{permittedNumberOfCallsInHalfOpenState}: 5 --- Allows a small "probe" traffic to verify recovery without risking a full system impact.
\end{itemize}

\subsection{Fallback Definition and RFC 9110}

The use of \textbf{HTTP 202 (Accepted)} as a fallback response is a deliberate architectural choice. According to \textbf{RFC 9110}, HTTP 202 indicates that the request has been accepted for processing, but the processing has not been completed. This aligns with a "scheduled payment" or "queue for later" pattern, maintaining semantic honesty while providing a graceful degradation path for the end user \cite{rfc9110}.

\subsection{State Transition Workflow}

The Circuit Breaker transitions through three primary states: \textbf{Closed} (all traffic flows), \textbf{Open} (traffic blocked/fallback), and \textbf{Half-Open} (controlled probing). Figure~\ref{fig:cb-workflow} illustrates this logic.

\begin{figure}[htbp]
\centering
\begin{minipage}{0.45\textwidth}
\raggedright
\small
\textbf{1. Closed} $\rightarrow$ \textbf{Open}: Triggered when $failureRate > 50\%$ in a 10-req window. \\
\textbf{2. Open} $\rightarrow$ \textbf{Half-Open}: Automatic after 10s wait duration. \\
\textbf{3. Half-Open}: Probes with 3 requests. \\
\textbf{4. Half-Open} $\rightarrow$ \textbf{Closed}: If all 3 probes succeed. \\
\textbf{5. Half-Open} $\rightarrow$ \textbf{Open}: If any probe fails.
\end{minipage}
\caption{Circuit Breaker State Machine Workflow Logic.}
\label{fig:cb-workflow}
\end{figure}

\subsection{Test Scenarios}

Five realistic failure scenarios were designed using Grafana k6 (Table~\ref{tab:scenarios}):

\begin{table}[H]
\centering
\caption{Test Scenario Characteristics}
\label{tab:scenarios}
\begin{tabular}{lccc}
\toprule
\textbf{Scenario} & \textbf{Duration} & \textbf{VUs} & \textbf{Failure Pattern} \\
\midrule
Catastrophe & 13min & 50-150 & 100\% failure for 5min \\
Degradation & 13min & 100-200 & 5\%→50\% gradual \\
Bursts & 13min & 100-200 & 3×(100\% for 1min) \\
Unavailability & 9min & 80-200 & 75\% offline \\
Normal & 10min & 100 & 100\% healthy \\
\bottomrule
\end{tabular}
\end{table}

\subsection{Metrics and Statistical Analysis}

Collected metrics include: http\_reqs (throughput), http\_req\_duration\{p(95)\} (latency percentile), and http\_req\_failed (error rate).

To reflect user-visible continuity of service, we define \textbf{Perceived Availability} as the fraction of requests that result in either a successful outcome (HTTP 200/201) or a graceful degradation outcome delivered through the fallback mechanism (HTTP 202). Following HTTP semantics, HTTP 202 (Accepted) indicates that the request has been accepted for processing, which is suitable for our “scheduled payment” fallback \cite{rfc9110}.

\[
A_p = \frac{n_{200} + n_{201} + n_{202}}{n_{\text{total}}}
\]

Statistical validation employed: Student's t-test for comparing V1 vs V2, ANOVA for three-group comparison (V1, V2, V3), and Cohen's d for effect size quantification.

\chapter{Detalhes da Implementação dos Serviços}
\label{cap:implementacao}

Este capítulo apresenta a documentação técnica das três versões do serviço de pagamento desenvolvidas para o experimento, detalhando a API, a arquitetura e as configurações de resiliência.

\section{Visão Geral}

\subsection{Arquitetura de Alto Nível}

O fluxo principal da aplicação consiste em:

\begin{enumerate}
    \item O Cliente (k6, usuário ou outro sistema) realiza uma chamada HTTP ao \textbf{serviço de pagamento}.
    \item O serviço de pagamento comunica-se com o \textbf{serviço adquirente} via \textbf{OpenFeign}.
    \item O adquirente simula comportamentos (normal, latência ou falha) controlados pelo parâmetro de consulta \texttt{modo}.
\end{enumerate}

A integração com o adquirente ocorre através da chamada \texttt{POST /autorizar?modo=...}. Os endereços de host e porta são configurados estaticamente no cliente Feign, apontando para \texttt{http://servico-adquirente:8081}, o que facilita a execução em ambiente Docker Compose.

\subsection{Comparativo das Versões}

A Tabela \ref{tab:comparativo-versoes} resume as características das três versões implementadas.

\begin{table}[H]
\centering
\caption{Comparativo das Versões do Serviço de Pagamento}
\label{tab:comparativo-versoes}
\begin{tabular}{@{}lp{3cm}p{4cm}p{4cm}@{}}
\toprule
\textbf{Versão} & \textbf{Objetivo} & \textbf{Resiliência} & \textbf{Observabilidade} \\ \midrule
v1 & Baseline (controle) & Sem Circuit Breaker ou Retry & Health (Actuator) \\
v2 & Circuit Breaker & \texttt{@CircuitBreaker} + fallback (202) & Health, CB endpoints, Métricas, Prometheus \\
v3 & Retry & \texttt{@Retry} + fallback após esgotar tentativas & Health, Retry endpoints, Métricas, Prometheus \\ \bottomrule
\end{tabular}
\end{table}

\section{API HTTP}

A API é comum às três versões e expõe o endpoint de pagamento.

\subsection{Endpoint: POST /pagar}

\begin{itemize}
    \item \textbf{Path:} \texttt{/pagar}
    \item \textbf{Método:} \texttt{POST}
    \item \textbf{Query params:} \texttt{modo} (valores: \texttt{normal}, \texttt{latencia}, \texttt{falha})
\end{itemize}

O corpo da requisição espera um JSON com os campos \texttt{amount}, \texttt{payment\_method} e \texttt{customer\_id}. O Código \ref{lst:curl-pagar} demonstra um exemplo de chamada via cURL.

\begin{lstlisting}[language=bash, caption={Exemplo de chamada cURL para o endpoint de pagamento}, label={lst:curl-pagar}]
curl -i -X POST "http://localhost:8080/pagar?modo=normal" \
  -H 'Content-Type: application/json' \
  -d '{
        "amount": 100.00,
        "payment_method": "credit_card",
        "customer_id": "customer-123",
        "order_id": "order-999"
      }'
\end{lstlisting}

\subsection{Semântica das Respostas}

\begin{itemize}
    \item \textbf{v1 (Baseline):} Retorna \texttt{200 OK} em sucesso ou \texttt{500 Internal Server Error} em caso de falha do adquirente.
    \item \textbf{v2 (Circuit Breaker):} Retorna \texttt{200 OK} em sucesso, \texttt{500} para erros mapeados e \texttt{202 Accepted} quando o fallback é acionado (circuito aberto ou falha), indicando degradação graciosa.
    \item \textbf{v3 (Retry):} Retorna \texttt{200 OK} em sucesso e \texttt{500} após esgotar todas as tentativas de retentativa. O fallback só é chamado ao final.
\end{itemize}

\section{Estrutura de Dados (DTOs)}

\subsection{PaymentRequest}

O DTO \texttt{PaymentRequest} mapeia os dados recebidos. O Código \ref{lst:payment-request} mostra a implementação do record Java.

\begin{lstlisting}[language=Java, caption={Implementação do DTO PaymentRequest}, label={lst:payment-request}]
public record PaymentRequest(
  BigDecimal amount,
  String paymentMethod,
  String customerId,
  Map<String, Object> additionalData
) {
  public static PaymentRequest fromMap(Map<String, Object> map) {
    BigDecimal amount = map.get("amount") != null
      ? new BigDecimal(map.get("amount").toString())
      : BigDecimal.ZERO;
    String paymentMethod = (String) map.getOrDefault("payment_method", "unknown");
    String customerId = (String) map.getOrDefault("customer_id", "anonymous");

    return new PaymentRequest(amount, paymentMethod, customerId, map);
  }

  public Map<String, Object> toMap() {
    return additionalData != null ? additionalData : Map.of(
      "amount", amount,
      "payment_method", paymentMethod,
      "customer_id", customerId
    );
  }
}
\end{lstlisting}

\section{Versão v1 — Baseline}

A versão v1 atua como grupo de controle, sem mecanismos avançados de resiliência.

\subsection{Implementação do Controller e Service}

O Controller (Código \ref{lst:v1-controller}) recebe a requisição e delega para o serviço.

\begin{lstlisting}[language=Java, caption={Controller da versão Baseline (v1)}, label={lst:v1-controller}]
@PostMapping(path = "/pagar")
public ResponseEntity<String> pagar(@RequestParam("modo") String modo,
                                    @RequestBody Map<String, Object> pagamento) {
    PaymentRequest request = PaymentRequest.fromMap(pagamento);
    PaymentResponse response = paymentService.processPayment(modo, request);
    return ResponseEntity.status(response.status()).body(response.message());
}
\end{lstlisting}

O Service (Código \ref{lst:v1-service}) realiza a chamada Feign direta.

\begin{lstlisting}[language=Java, caption={Service da versão Baseline (v1)}, label={lst:v1-service}]
ResponseEntity<String> response = acquirerClient.autorizarPagamento(modo, request.toMap());
if (response.getStatusCode().is5xxServerError()) {
    return PaymentResponse.failure("Erro do adquirente: " + response.getBody());
}
return PaymentResponse.success(response.getBody());
\end{lstlisting}

\subsection{Configuração}

A configuração no \texttt{application.yml} define timeouts básicos para o cliente Feign.

\begin{lstlisting}[language=yaml, caption={Configuração do Feign na v1}, label={lst:v1-config}]
server:
  port: 8080

feign:
  client:
    config:
      default:
        connectTimeout: 2000
        readTimeout: 2000
\end{lstlisting}

\section{Versão v2 — Circuit Breaker}

A versão v2 utiliza o Resilience4j para implementar o padrão Circuit Breaker.

\subsection{Implementação com Circuit Breaker}

O método de processamento é anotado com \texttt{@CircuitBreaker}, definindo um método de fallback (Código \ref{lst:v2-service}).

\begin{lstlisting}[language=Java, caption={Service com Circuit Breaker (v2)}, label={lst:v2-service}]
@CircuitBreaker(name = "adquirente-cb", fallbackMethod = "processPaymentFallback")
@Timed(value = "payment.processing.time")
public PaymentResponse processPayment(String modo, PaymentRequest request) {
    ResponseEntity<String> response = acquirerClient.autorizarPagamento(modo, request.toMap());
    if (response.getStatusCode() == HttpStatus.SERVICE_UNAVAILABLE) {
        return PaymentResponse.failure("Serviço adquirente indisponível: " + response.getBody());
    }
    return PaymentResponse.success(response.getBody());
}
\end{lstlisting}

O método de fallback (Código \ref{lst:v2-fallback}) trata a abertura do circuito retornando uma resposta de aceitação (202).

\begin{lstlisting}[language=Java, caption={Método de Fallback (v2)}, label={lst:v2-fallback}]
public PaymentResponse processPaymentFallback(String modo, PaymentRequest request, Throwable t) {
    if (t instanceof CallNotPermittedException) {
        return PaymentResponse.circuitBreakerOpen();
    }
    return PaymentResponse.fallback("Pagamento aceito para processamento posterior: " + t.getMessage());
}
\end{lstlisting}

\subsection{Configuração do Circuit Breaker}

O perfil \texttt{equilibrado} define os limiares de falha e janelas de tempo (Código \ref{lst:v2-config}).

\begin{lstlisting}[language=yaml, caption={Configuração do Circuit Breaker (Perfil Equilibrado)}, label={lst:v2-config}]
resilience4j:
  circuitbreaker:
    instances:
      adquirente-cb:
        failureRateThreshold: 50
        slidingWindowType: COUNT_BASED
        slidingWindowSize: 20
        minimumNumberOfCalls: 10
        waitDurationInOpenState: 10s
        permittedNumberOfCallsInHalfOpenState: 5
        slowCallDurationThreshold: 2000ms
        slowCallRateThreshold: 80
        automaticTransitionFromOpenToHalfOpenEnabled: true
\end{lstlisting}

\section{Versão v3 — Retry}

A versão v3 implementa retentativas com backoff exponencial.

\subsection{Implementação com Retry}

O serviço utiliza a anotação \texttt{@Retry} (Código \ref{lst:v3-service}).

\begin{lstlisting}[language=Java, caption={Service com Retry (v3)}, label={lst:v3-service}]
@Retry(name = "adquirente-retry", fallbackMethod = "processPaymentFallback")
@Timed(value = "payment.processing.time")
public PaymentResponse processPayment(String modo, PaymentRequest request) {
    ResponseEntity<String> response = acquirerClient.autorizarPagamento(modo, request.toMap());

    if (response.getStatusCode() == HttpStatus.SERVICE_UNAVAILABLE) {
        throw new RuntimeException("Serviço adquirente indisponível: " + response.getBody());
    }

    if (response.getStatusCode().is5xxServerError()) {
        throw new RuntimeException("Erro no serviço adquirente: " + response.getBody());
    }

    return PaymentResponse.success(response.getBody());
}
\end{lstlisting}

\subsection{Configuração do Retry}

As configurações de backoff exponencial são definidas no \texttt{application.yml} (Código \ref{lst:v3-config}).

\begin{lstlisting}[language=yaml, caption={Configuração do Retry (v3)}, label={lst:v3-config}]
resilience4j:
  retry:
    instances:
      adquirente-retry:
        maxAttempts: 3
        waitDuration: 500ms
        enableExponentialBackoff: true
        exponentialBackoffMultiplier: 2
        enableRandomizedWait: true
        randomizedWaitFactor: 0.5
        retryExceptions:
          - java.net.SocketTimeoutException
          - java.net.ConnectException
          - java.io.IOException
          - feign.FeignException$ServiceUnavailable
          - feign.FeignException$InternalServerError
\end{lstlisting}

\section{Execução com Docker Compose}

O ambiente é orquestrado via Docker Compose, mapeando as portas para cada versão do serviço (Código \ref{lst:docker-compose}).

\begin{lstlisting}[language=yaml, caption={Trecho do docker-compose.yml}, label={lst:docker-compose}]
  servico-pagamento-v2:
    ports:
      - "8082:8080"

  servico-pagamento-v3:
    ports:
      - "8083:8080"
\end{lstlisting}

\section{Results and Discussion}
\label{sec:results}

Load tests were executed using k6 and Docker Compose. Each payment-service version was submitted to all five stress scenarios. Results were evaluated against defined thresholds and analyzed in terms of success rate, response time, and fallback contribution.

\subsection{Consolidated Results Overview}

Table~\ref{tab:consolidated} presents the consolidated comparison between V1 (Baseline) and V2 (Circuit Breaker).

\begin{table}[!t]
\centering
\caption{Consolidated Comparison: V1 vs V2 by Scenario}
\label{tab:consolidated}
\small
\begin{tabular}{lcccc}
\toprule
Scenario & V1 & V2 & Fallback & Fail. Red. \\
\midrule
Catastrophe & 35.7\% & \textbf{95.1\%} & 62.1\% & -92.3\% \\
Degradation & 75.4\% & \textbf{95.4\%} & 64.7\% & -81.4\% \\
Unavail. & 10.5\% & \textbf{99.6\%} & 99.1\% & -99.5\% \\
Normal & 100.0\% & 100.0\% & 0.0\% & 0.0\% \\
Bursts & 63.0\% & \textbf{96.7\%} & 34.6\% & -91.0\% \\
\bottomrule
\end{tabular}
\end{table}

\textbf{Key Findings:}
\begin{itemize}
    \item V2 demonstrated gains of \textbf{+20pp to +89pp} in success rate across all failure scenarios;
    \item In Extreme Unavailability, V2 transformed a system with only 10.5\% success into one with 99.6\%—a \textbf{9.5x improvement};
    \item The fallback mechanism (HTTP 202) accounted for up to 99.1\% of successful responses in worst scenarios;
    \item In Normal scenario, no difference between versions confirms CB introduces \textbf{no overhead} in healthy conditions.
\end{itemize}

\subsection{V3 (Retry) Analysis}

Table~\ref{tab:v3comparison} includes V3 (Retry with exponential backoff) for comparison.

\begin{table}[!t]
\centering
\caption{Detailed Comparison: V1 vs V2 vs V3}
\label{tab:v3comparison}
\small
\begin{tabular}{lrrr}
\toprule
Scenario & V1 & V2 (CB) & V3 (Retry) \\
\midrule
Catastrophe & 35.7\% & \textbf{95.1\%} & 52.8\% \\
Degradation & 75.4\% & \textbf{95.4\%} & 76.4\% \\
Unavail. & 10.5\% & \textbf{99.6\%} & 15.7\% \\
Normal & 100.0\% & 100.0\% & 100.0\% \\
Bursts & 63.0\% & \textbf{96.7\%} & 77.7\% \\
\bottomrule
\end{tabular}
\end{table}

\vspace{1em}

\textbf{Critical Observations on Retry:}
\begin{enumerate}
    \item Retry does NOT improve availability in persistent failures;
    \item Retry increases latency due to retry attempts;
    \item Retry can amplify problems by tripling load on overloaded services.
\end{enumerate}

\subsection{Statistical Analysis}

Table~\ref{tab:statistics} presents formal statistical validation.

\begin{table}[!t]
\centering
\caption{Statistical Analysis: V1 vs V2}
\label{tab:statistics}
\small
\begin{tabular}{lr}
\toprule
Metric & Value \\
\midrule
t-test (p-value) & p $<$ 0.0001 \\
Cohen's d & 1.078 (\textbf{large}) \\
\midrule
ANOVA F-statistic & 546.79 (p $<$ 0.0001) \\
Eta-squared ($\eta^2$) & 0.267 (large) \\
\midrule
95\% CI (V1) & [495.86; 508.01] ms \\
95\% CI (V2) & [400.72; 410.62] ms \\
95\% CI (V3) & [543.38; 558.02] ms \\
\bottomrule
\end{tabular}
\end{table}

\vspace{0.5em}

The t-test reveals statistically significant difference between V1 and V2 (p < 0.0001). Cohen's d (d = 1.078) classifies the effect size as \textbf{large}, confirming substantial practical relevance. ANOVA confirmed significant difference among all three groups (F = 546.79, p < 0.0001).

\subsection{Quantified Impact}

The Circuit Breaker provided significant gains across all failure scenarios:
\begin{itemize}
    \item \textbf{Catastrophe:} +59.3pp (35.7\% $\rightarrow$ 95.1\%)
    \item \textbf{Degradation:} +20.0pp (75.4\% $\rightarrow$ 95.4\%)
    \item \textbf{Unavailability:} +89.1pp (10.5\% $\rightarrow$ 99.6\%)
    \item \textbf{Bursts:} +33.6pp (63.0\% $\rightarrow$ 96.7\%)
\end{itemize}

\subsection{Detailed Analysis: Bursts and Catastrophe}

Scenarios of \textbf{Intermittent Bursts} and \textbf{Catastrophic Failure} deserve special attention as they demonstrate the most significant Circuit Breaker benefits. Figure~\ref{fig:v1v2comparison} shows the direct comparison.

\begin{figure}[H]
\centering
\includegraphics[width=0.45\textwidth]{images/01_v1_v2_success_rate_comparison.png}
\caption{Success Rate Comparison: V1 vs V2 in Bursts and Catastrophe}
\label{fig:v1v2comparison}
\end{figure}

\textbf{Key Findings:}
\begin{itemize}
    \item In \textbf{Bursts}, V1 achieved only 63.0\% success while V2 reached \textbf{96.7\%}, a gain of \textbf{+33.6pp}.
    \item In \textbf{Catastrophe}, V1 registered 35.7\% success (system nearly unusable) while V2 maintained \textbf{95.1\%}, a gain of \textbf{+59.3pp}.
    \item Failure reduction exceeded \textbf{91\%} in both scenarios.
\end{itemize}

The Circuit Breaker achieves these results by transforming HTTP 500 errors into HTTP 202 (fallback) responses. In the Catastrophe scenario, 62.1\% of V2 responses came from fallback, effectively converting fatal errors into meaningful responses for end users.

\begin{figure}[H]
\centering
\includegraphics[width=0.48\textwidth]{images/05_combined_summary.png}
\caption{Consolidated Impact Analysis: Bursts and Catastrophe}
\label{fig:consolidated}
\end{figure}

These results validate three fundamental characteristics:
(1) \textbf{Fail-Fast with Graceful Degradation} --- the system returns meaningful alternative responses immediately;
(2) \textbf{Elasticity} --- the CB transitions dynamically between states based on dependency health;
(3) \textbf{Resource Protection} --- threads are released immediately during failures, preventing thread pool starvation.

\textbf{Sliding Window Detection Mechanism:}
A notable aspect is the CB's ability to ``anticipate'' failures through its sliding window mechanism. The CB monitors the last 10 requests and opens when failure rate exceeds 50\%. In the Catastrophe scenario, fallback rate (62.1\%) closely matched V1's actual failure rate (64.3\%), yielding a \textbf{96.6\% coverage ratio}. In Bursts, the CB demonstrated elasticity by transitioning states dynamically: fallback rate (34.6\%) matched V1 failures (37.0\%) with 93.5\% coverage, while maintaining direct success rate (62.0\%) nearly identical to V1 (63.0\%). This indicates the CB did not block requests unnecessarily during healthy periods, but correctly activated fallback during actual failures --- achieving intelligent fail-fast behavior through the half-open state mechanism that periodically probes dependency health.

\chapter{Conclusão}
\label{cap:conclusao}

\section{Revisão dos Objetivos e do Problema}
Este trabalho se propôs a investigar a fragilidade da comunicação síncrona em microsserviços, especificamente o risco de falhas em cascata em um sistema de pagamentos. O objetivo foi avaliar quantitativamente o impacto do padrão Circuit Breaker no desempenho e resiliência, usando um experimento prático e reprodutível com \texttt{Docker} e \texttt{k6}.

\section{Síntese dos Resultados}
Os resultados experimentais foram conclusivos e demonstraram de forma inequívoca o valor do padrão Circuit Breaker, bem como as limitações do padrão Retry quando usado isoladamente:

\begin{itemize}
    \item \textbf{Disponibilidade Total:} A V2 (Circuit Breaker) foi a \textbf{única} versão a alcançar 100\% de disponibilidade, eliminando completamente falhas visíveis ao usuário através do mecanismo de fallback.
    \item \textbf{Superioridade sobre Retry:} A V3 (Retry com backoff exponencial) apresentou taxa de sucesso idêntica à V1 (89,99\% vs 89,97\%), demonstrando que \textbf{Retry sozinho não melhora disponibilidade} em cenários de falha persistente.
    \item \textbf{Melhoria de Performance:} O tempo médio de resposta da V2 foi de 179ms contra 534ms da V1 (\textbf{-66,5\%}), enquanto a V3 teve o pior desempenho com 722ms (\textbf{+35\%} em relação à V1) devido às retentativas.
    \item \textbf{Aumento de Throughput:} A V2 processou 289 req/s contra 222 req/s da V1 (\textbf{+30\%}) e 198 req/s da V3 (\textbf{+46\%}).
    \item \textbf{Análise Estatística:} O teste de Mann-Whitney confirmou diferença significativa (p $<$ 0,001), e o Cliff's Delta ($\delta = 0,594$) indica \textbf{effect size grande}, confirmando relevância prática substancial e não resultado do acaso.
    \item \textbf{Elasticidade:} O CB demonstrou capacidade de transicionar dinamicamente entre estados conforme o comportamento da dependência, abrindo e fechando o circuito conforme necessário.
\end{itemize}

A arquitetura Baseline (V1) e a versão com Retry (V3) provaram ser \textbf{inadequadas para produção} em sistemas de missão crítica, enquanto a V2 (Circuit Breaker) demonstrou robustez excepcional, protegendo o \texttt{servico-pagamento} e garantindo disponibilidade percebida pelo usuário através da degradação graciosa (HTTP 202). Os resultados são particularmente relevantes para sistemas financeiros, e-commerce e qualquer domínio onde a disponibilidade é requisito crítico.

\section{Contribuições do Trabalho}
Este TCC contribui para a literatura ao fornecer:
\begin{enumerate}
    \item \textbf{Evidência Empírica Robusta:} Dados quantitativos que demonstram o impacto real do Circuit Breaker em cenários realistas de falha, com mais de 1.278.000 requisições analisadas (V1 + V2 + V3).
    \item \textbf{Comparação entre Padrões:} Primeira análise comparativa entre Circuit Breaker e Retry isolado, demonstrando que Retry \textbf{não substitui} o CB.
    \item \textbf{Análise Estatística Rigorosa:} Aplicação de testes não-paramétricos (Mann-Whitney U, Kolmogorov-Smirnov) e medidas de effect size (Cliff's Delta = 0,594 --- \textbf{grande}) que confirmam relevância prática substancial.
    \item \textbf{Metodologia Reprodutível:} Um framework de testes com Docker e k6 que pode ser replicado para avaliar outros padrões de resiliência.
    \item \textbf{Quantificação de Benefícios:} Demonstração de melhorias de 100\% de disponibilidade, 66,5\% em tempo médio de resposta e 30\% em throughput.
\end{enumerate}

\section{Limitações do Estudo}
Apesar dos resultados expressivos, este trabalho apresenta limitações que devem ser consideradas na interpretação dos achados:

\begin{enumerate}
    \item \textbf{POC Simplificada:} O sistema experimental é uma Prova de Conceito intencionalmente rudimentar, sem banco de dados, cache, autenticação ou lógica de negócio complexa. Sistemas de produção reais possuem mais variáveis que podem interagir com o comportamento do Circuit Breaker.
    
    \item \textbf{Ambiente Local:} Os experimentos foram executados em uma única máquina, sem latência de rede real entre serviços. Em ambientes distribuídos (multi-datacenter, cloud), a latência adicional pode influenciar os resultados.
    
    \item \textbf{Carga Sintética:} O k6 gera tráfego sintético com padrões uniformes. Tráfego real apresenta características mais complexas: rajadas imprevisíveis, sazonalidade e correlação entre requisições.
    
    \item \textbf{Serviço Adquirente Mockado:} O \texttt{servico-adquirente} foi implementado com falhas controladas e determinísticas. Em produção, falhas são frequentemente parciais e imprevisíveis.
    
    \item \textbf{Execução Única:} Cada cenário foi executado uma vez por versão. Múltiplas execuções permitiriam calcular intervalos de confiança e validar reprodutibilidade.
    
    \item \textbf{Configuração Fixa do CB:} Apenas uma configuração do Circuit Breaker foi testada. Diferentes parametrizações podem resultar em comportamentos distintos.
    
    \item \textbf{Domínio Específico:} Embora o contexto de pagamentos seja generalizável, outros domínios podem ter requisitos específicos (ex: latência ultra-baixa em trading) que influenciam a aplicabilidade dos resultados.
\end{enumerate}

\section{Ameaças à Validade}
Identificamos as seguintes ameaças à validade dos resultados:

\textbf{Validade Interna:}
\begin{itemize}
    \item \textbf{Variabilidade do ambiente:} Processos em segundo plano no sistema operacional, garbage collection da JVM e contenção de recursos do Docker podem introduzir variância nos resultados.
    \item \textbf{Warm-up da JVM:} A compilação JIT (Just-In-Time) pode afetar os primeiros minutos de cada teste. Mitigamos isso com fases de aquecimento nos scripts k6.
    \item \textbf{Ordem de execução:} Os testes foram executados sequencialmente; efeitos de ordem (ex: fragmentação de memória) não foram controlados.
\end{itemize}

\textbf{Validade Externa:}
\begin{itemize}
    \item \textbf{Generalização:} Os resultados são específicos para o domínio de pagamentos e a stack tecnológica utilizada (Java/Spring). Outros domínios e tecnologias podem apresentar comportamentos diferentes.
    \item \textbf{Escala:} O experimento utilizou até 200 VUs (usuários virtuais). Sistemas de produção podem enfrentar milhares de requisições simultâneas, alterando a dinâmica de contenção.
    \item \textbf{Complexidade arquitetural:} O experimento envolveu apenas dois serviços. Arquiteturas reais com dezenas de microsserviços introduzem cadeias de dependência mais complexas.
\end{itemize}

\textbf{Validade de Construção:}
\begin{itemize}
    \item \textbf{Definição de sucesso:} Consideramos HTTP 200 e HTTP 202 (fallback) como sucesso. Em contextos onde o fallback não é aceitável pelo negócio, a interpretação dos resultados seria diferente.
    \item \textbf{Métricas selecionadas:} Focamos em taxa de sucesso, latência e throughput. Outras métricas (uso de CPU, memória, conexões abertas) poderiam revelar aspectos adicionais.
\end{itemize}

\section{Trabalhos Futuros}
Como trabalho futuro, sugere-se:
\begin{enumerate}
    \item \textbf{Comparação com Outros Padrões:} Expandir o experimento para comparar o Circuit Breaker com outros padrões de resiliência como Retry (com backoff exponencial), Bulkhead (isolamento de threads) e Rate Limiter, avaliando tanto o uso isolado quanto a composição destes padrões.
    
    \item \textbf{Análise Paramétrica:} Investigar sistematicamente o impacto de diferentes configurações do Circuit Breaker (ex: \texttt{slidingWindowSize}, \texttt{failureRateThreshold}, \texttt{waitDurationInOpenState}), identificando configurações ótimas para diferentes perfis de carga.
    
    \item \textbf{Cenários de Múltiplas Dependências:} Avaliar o comportamento do CB em arquiteturas com múltiplos serviços dependentes, investigando estratégias de Circuit Breaker por dependência versus Circuit Breaker global.
    
    \item \textbf{Comunicação Assíncrona:} Comparar os resultados com arquiteturas baseadas em mensageria (Apache Kafka, RabbitMQ), avaliando os trade-offs entre comunicação síncrona protegida por CB e comunicação assíncrona nativa.
    
    \item \textbf{Ambiente Cloud Distribuído:} Replicar os experimentos em ambiente cloud (AWS, GCP ou Azure) com serviços distribuídos geograficamente, introduzindo latência de rede real e avaliando o comportamento do CB em cenários de particionamento de rede.
    
    \item \textbf{Chaos Engineering:} Integrar ferramentas de Chaos Engineering (ex: Chaos Monkey, Litmus) para injeção de falhas aleatórias e contínuas, validando a robustez do Circuit Breaker em condições mais realistas e imprevisíveis.
    
    \item \textbf{Observabilidade Avançada:} Implementar distributed tracing (Jaeger, Zipkin) para correlacionar o estado do Circuit Breaker com traces de requisições, facilitando a análise de causa raiz em cenários complexos.
    
    \item \textbf{Machine Learning para Tuning:} Explorar o uso de algoritmos de aprendizado de máquina para ajuste dinâmico dos parâmetros do Circuit Breaker com base em padrões históricos de tráfego e falhas.
    
    \item \textbf{Estudo Longitudinal:} Conduzir um estudo de longo prazo em ambiente de produção para avaliar a eficácia do Circuit Breaker ao longo de meses, capturando eventos reais de degradação e falha.
\end{enumerate}

 % O conteúdo do usuário vai aqui

\postextual
\bibliography{references}
\end{document}
