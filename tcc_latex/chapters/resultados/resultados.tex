\chapter{Resultados e Discussão}
\label{cap:resultados}

\section{Introdução ao Capítulo}
Os testes de carga foram executados usando \texttt{k6} e \texttt{Docker Compose}. Cada versão do \texttt{servico-pagamento} (V1-Baseline, V2-CircuitBreaker) foi submetida aos quatro cenários de estresse (Falha Catastrófica, Degradação Gradual, Rajadas Intermitentes e Indisponibilidade Extrema). Os resultados foram avaliados contra os \texttt{thresholds} (limites de desempenho) definidos nos scripts e analisados em termos de taxa de sucesso, tempo de resposta e contribuição do mecanismo de fallback.

\section{Visão Geral Consolidada dos Resultados}

Este trabalho avaliou três versões do serviço de pagamento: V1 (Baseline sem resiliência), V2 (Circuit Breaker com Resilience4j) e V3 (Retry com Backoff Exponencial). Os resultados demonstram diferenças significativas entre as abordagens.

\begin{table}[H]
\centering
\caption{Resumo Consolidado: Comparação V1 vs V2 vs V3 - Métricas Gerais}
\label{tab:resumo-consolidado}
\begin{tabular}{lccc}
\toprule
Métrica & V1 (Baseline) & V2 (Circuit Breaker) & V3 (Retry) \\
\midrule
Requisições Totais & 400.647 & 521.209 & 356.979 \\
Disponibilidade & 89,97\% & \textbf{100\%} & 89,99\% \\
Taxa de Sucesso Real & 89,97\% & 28,96\% & 89,99\% \\
Taxa de Fallback & 0\% & 71,04\% & 0\% \\
Taxa de Falha & 10,03\% & \textbf{0\%} & 10,00\% \\
Tempo Médio & 534 ms & \textbf{179 ms} & 722 ms \\
Mediana & 38 ms & \textbf{3 ms} & 84 ms \\
P95 & 2.771 ms & \textbf{2.245 ms} & 2.808 ms \\
Throughput & 222 req/s & \textbf{289 req/s} & 198 req/s \\
\bottomrule
\end{tabular}
\end{table}

\textbf{Observações Importantes:}
\begin{itemize}
    \item A V2 (Circuit Breaker) é a \textbf{única} que alcançou 100\% de disponibilidade, eliminando completamente falhas visíveis ao usuário.
    \item A V3 (Retry) apresentou taxa de sucesso similar à V1, mas com \textbf{tempo médio 35\% maior} devido às retentativas.
    \item O throughput da V2 foi \textbf{30\% superior} à V1 e \textbf{46\% superior} à V3.
\end{itemize}

A Tabela 2 apresenta os ganhos por cenário de teste específico, evidenciando o impacto transformador do Circuit Breaker em cenários de alta indisponibilidade.

\begin{figure}[H]
\centering
\includegraphics[width=0.85\textwidth]{images/01_success_rates_comparison.png}
\caption{Comparação de Taxas de Sucesso: V1 (Baseline) vs V2 (Circuit Breaker)}
\label{fig:success_rates}
\end{figure}

\section{Análise do Cenário 1: Falha Catastrófica}

Neste cenário, a API externa ficou 100\% indisponível durante 5 minutos consecutivos, simulando uma queda total do servidor.

\textbf{Comportamento Observado:}
\begin{itemize}
    \item \textbf{V1 (Baseline):} Durante o período de catástrofe, todas as requisições aguardaram o timeout (2s) antes de falhar, consumindo threads e degradando o sistema. A taxa de sucesso geral foi de 90,0\%. O downtime total foi de 78 segundos.
    \item \textbf{V2 (Circuit Breaker):} O CB detectou as falhas em menos de 10 segundos, abriu o circuito e passou a retornar respostas via fallback (HTTP 202) em menos de 100ms. A taxa de sucesso subiu para 94,5\%, com 59\% das respostas sendo fallbacks. O downtime foi reduzido para 43 segundos.
\end{itemize}

\textbf{Métricas de Performance:}
\begin{itemize}
    \item \textbf{Melhoria no Tempo de Resposta:} 60\% de redução no tempo médio
    \item \textbf{Redução de Falhas:} 44,8\%
    \item \textbf{Redução de Downtime:} 45\% (de 78s para 43s)
\end{itemize}

\begin{figure}[H]
\centering
\includegraphics[width=0.85\textwidth]{images/07_catastrofe_timeline.png}
\caption{Timeline do Cenário de Falha Catastrófica: Transição de Estados}
\label{fig:catastrofe_timeline}
\end{figure}

\textbf{Impacto Positivo do Circuit Breaker:}
\begin{enumerate}
    \item \textbf{Proteção contra Exaustão de Recursos:} Enquanto a V1 mantinha threads bloqueadas aguardando timeout, a V2 liberava threads imediatamente após a abertura do circuito.
    \item \textbf{Tempo de Resposta Previsível:} Média de resposta da V2 durante catástrofe: ~85ms (fallback). V1: ~2000ms (timeout).
    \item \textbf{Degradação Graciosa:} Usuários receberam HTTP 202 informando que o pagamento foi agendado, em vez de erro HTTP 500.
\end{enumerate}

\section{Análise do Cenário 2: Degradação Gradual}

Este cenário simula uma situação comum em produção: um serviço que começa saudável mas degrada progressivamente.

\textbf{Comportamento Observado:}
\begin{itemize}
    \item \textbf{V1 (Baseline):} A degradação afetou uniformemente todas as requisições. Taxa de sucesso: 94.72\%.
    \item \textbf{V2 (Circuit Breaker):} O CB detectou o aumento na taxa de falhas e latência, mantendo o sistema estável. Taxa de sucesso: 94.94\%, com fallback praticamente não acionado (0.0\%).
\end{itemize}

\textbf{Insight:} Neste cenário, o ganho em taxa de sucesso foi marginal (+0,2pp), pois a degradação não foi severa o suficiente para acionar o CB consistentemente. Entretanto, a V2 ainda demonstrou ligeira melhoria no tempo de resposta (0,44\% mais rápido) e redução de 4,2\% nas falhas, confirmando que o CB \textbf{não introduz overhead} em cenários de degradação moderada e ainda oferece proteção incremental.

\section{Análise do Cenário 3: Rajadas Intermitentes}

O cenário de rajadas é particularmente desafiador, pois exige que o sistema reaja rapidamente a mudanças de estado.

\textbf{Comportamento Observado:}
\begin{itemize}
    \item \textbf{V1 (Baseline):} Durante cada rajada de 1 minuto, 100\% das requisições falharam. Nos períodos normais, o sistema se recuperou. Taxa geral: 94,9\%.
    \item \textbf{V2 (Circuit Breaker):} O CB abriu durante as rajadas e fechou nos períodos normais, demonstrando \textbf{elasticidade}. Taxa geral: 95,2\%, com 10,2\% de fallbacks. Melhoria de 10,8\% no tempo de resposta médio.
\end{itemize}

\textbf{Elasticidade do Circuit Breaker:} A capacidade do CB de transicionar entre estados (Fechado $\rightarrow$ Aberto $\rightarrow$ Semiaberto $\rightarrow$ Fechado) foi validada neste cenário. O tempo médio para abertura do circuito foi de ~8 segundos após o início de cada rajada. A redução de 5,8\% nas falhas demonstra a proteção oferecida mesmo em cenários de instabilidade.

\section{Análise do Cenário 4: Indisponibilidade Extrema (75\% OFF)}

Este é o cenário mais extremo e onde o Circuit Breaker demonstrou seu valor máximo.

\begin{table}[H]
\centering
\caption{Resultados Detalhados: Indisponibilidade Extrema}
\label{tab:indisponibilidade-detalhes}
\begin{tabular}{lcc}
\toprule
Métrica & V1 (Baseline) & V2 (Circuit Breaker) \\
\midrule
Taxa de Sucesso Total & 10,14\% & 97,08\% \\
Requisições via Fallback & N/A & 92,80\% \\
Taxa de Falha Real & 89,86\% & 2,91\% \\
Redução de Falhas & \textemdash & \textbf{96,77\%} \\
\midrule
Downtime (segundos) & 487,4s & 15,8s \\
Disponibilidade Efetiva & 10,1\% & 97,1\% \\
Melhoria P95 & \textemdash & \textbf{95,6\%} \\
Melhoria Tempo Médio & \textemdash & \textbf{74,6\%} \\
\bottomrule
\end{tabular}
\end{table}

\textbf{Comportamento Observado:}
\begin{itemize}
    \item \textbf{V1 (Baseline):} Com 75\% de indisponibilidade, o sistema se tornou praticamente inutilizável. Apenas 10,14\% das requisições foram bem-sucedidas, caracterizando \textbf{falha catastrófica do serviço}. O downtime efetivo foi de \textbf{487 segundos} (mais de 8 minutos).
    \item \textbf{V2 (Circuit Breaker):} O CB manteve o circuito aberto durante os períodos de indisponibilidade, servindo fallbacks e preservando a disponibilidade percebida pelo usuário em 97,08\%. O downtime foi reduzido para apenas \textbf{16 segundos}.
\end{itemize}

\textbf{Impacto Transformador:} Este cenário demonstra que o Circuit Breaker pode transformar um sistema \textbf{completamente inutilizável} (10\% de sucesso) em um sistema \textbf{altamente disponível} (97\% de sucesso), representando uma melhoria de quase \textbf{10x na disponibilidade percebida}.

\begin{figure}[H]
\centering
\includegraphics[width=0.85\textwidth]{images/11_downtime_availability.png}
\caption{Comparação de Disponibilidade e Downtime por Cenário}
\label{fig:downtime_availability}
\end{figure}

\begin{figure}[H]
\centering
\includegraphics[width=0.85\textwidth]{images/02_failure_reduction.png}
\caption{Redução de Falhas: Impacto do Circuit Breaker por Cenário}
\label{fig:failure_reduction}
\end{figure}

\begin{figure}[H]
\centering
\includegraphics[width=0.85\textwidth]{images/09_avg_response_times.png}
\caption{Comparação de Tempos de Resposta Médios: V1 vs V2 por Cenário}
\label{fig:avg_response_times}
\end{figure}

\section{Análise de Tempos de Resposta}

\begin{figure}[H]
\centering
\includegraphics[width=0.85\textwidth]{images/03_response_time_percentiles.png}
\caption{Distribuição de Tempos de Resposta (Percentis) por Cenário}
\label{fig:response_percentiles}
\end{figure}

A análise dos percentis de tempo de resposta revela outro benefício crucial do Circuit Breaker: \textbf{previsibilidade}. Enquanto a V1 apresentou alta variância nos tempos de resposta (de 50ms a 2000ms dependendo do estado da dependência), a V2 manteve tempos consistentemente baixos devido ao mecanismo de fail-fast.

\begin{figure}[H]
\centering
\includegraphics[width=0.85\textwidth]{images/distributions.png}
\caption{Distribuição Estatística dos Tempos de Resposta: V1 vs V2}
\label{fig:distributions}
\end{figure}

A Figura \ref{fig:distributions} apresenta a distribuição estatística dos tempos de resposta para ambas as versões. Note a maior concentração de requisições em tempos baixos na V2, com mediana de apenas 35,69ms comparada aos 160,56ms da V1.

\begin{figure}[H]
\centering
\includegraphics[width=0.85\textwidth]{images/distribution_boxplot.png}
\caption{Distribuição do Tempo de Resposta (Box Plot) por Cenário}
\label{fig:distribution-boxplot}
\end{figure}

\subsection{Análise de Consistência e Variabilidade}

Além das métricas tradicionais de latência, foi calculado o \textbf{Coeficiente de Variação (CV)} para cada cenário, definido como a razão entre o desvio padrão e a média do tempo de resposta. Esta métrica quantifica a \textbf{previsibilidade} do sistema — valores menores indicam comportamento mais consistente.

A análise revelou que a V2 (Circuit Breaker) apresenta CV significativamente menor em cenários adversos:

\begin{itemize}
    \item \textbf{Cenário de Indisponibilidade Extrema:} V1 apresentou CV de 1,48 (alta variabilidade), enquanto V2 alcançou CV de 0,87 (moderada variabilidade) — uma redução de 41\% na dispersão relativa.
    \item \textbf{Cenário de Falha Catastrófica:} A V2 manteve tempos de resposta mais consistentes durante o período de fallback, com CV de 0,92 versus 1,21 da V1.
    \item \textbf{Cenário de Rajadas:} Ambas as versões apresentaram variabilidade similar em períodos normais, mas a V2 demonstrou recuperação mais consistente após as rajadas.
\end{itemize}

A redução do CV na V2 tem implicações práticas importantes: sistemas mais previsíveis facilitam o dimensionamento de recursos, melhoram a experiência do usuário (que não enfrenta oscilações de latência) e simplificam a definição de SLAs baseados em percentis.


\begin{figure}[H]
\centering
\includegraphics[width=0.85\textwidth]{images/05_status_distribution.png}
\caption{Distribuição de Códigos de Status HTTP por Cenário}
\label{fig:status_distribution}
\end{figure}

A Figura \ref{fig:status_distribution} mostra a distribuição de códigos de status HTTP para cada cenário. Observa-se claramente que a V2 substitui erros HTTP 500 por respostas HTTP 202 (fallback), mantendo o sistema funcional para o usuário final.

\begin{figure}[H]
\centering
\includegraphics[width=0.85\textwidth]{images/10_error_rates.png}
\caption{Comparação de Taxas de Erro: V1 vs V2 por Cenário}
\label{fig:error_rates}
\end{figure}

A Figura \ref{fig:error_rates} evidencia a redução dramática nas taxas de erro proporcionada pelo Circuit Breaker, especialmente no cenário de indisponibilidade extrema onde a taxa de erro caiu de 89,9\% para apenas 2,9\%.

\section{Contribuição do Mecanismo de Fallback}

\begin{figure}[H]
\centering
\includegraphics[width=0.85\textwidth]{images/08_fallback_contribution.png}
\caption{Contribuição do Fallback para a Taxa de Sucesso da V2}
\label{fig:fallback_contribution}
\end{figure}

O gráfico da Figura \ref{fig:fallback_contribution} ilustra como o fallback contribuiu para a taxa de sucesso em cada cenário. No cenário de Indisponibilidade Extrema, \textbf{92.80\%} das respostas bem-sucedidas vieram do fallback, demonstrando que o sistema manteve sua utilidade mesmo com a dependência quase completamente indisponível.

\begin{figure}[H]
\centering
\includegraphics[width=0.85\textwidth]{images/timeline_comparison.png}
\caption{Comparação Temporal de Tempos de Resposta: V1 vs V2}
\label{fig:timeline_comparison}
\end{figure}

A Figura \ref{fig:timeline_comparison} apresenta a evolução temporal dos tempos de resposta ao longo de todos os cenários de teste. Observa-se que ambas as versões apresentam picos de latência nos mesmos momentos (correspondentes às fases de estresse), porém a V2 demonstra recuperação mais rápida devido ao mecanismo de fallback.

\section{Análise de Throughput}

\begin{figure}[H]
\centering
\includegraphics[width=0.85\textwidth]{images/04_throughput_comparison.png}
\caption{Comparação de Throughput (Requisições/segundo) entre V1 e V2}
\label{fig:throughput}
\end{figure}

Além da taxa de sucesso, o throughput (vazão) é uma métrica crítica. A V2 manteve throughput superior em cenários de falha porque não desperdiçou threads aguardando timeouts. Isso se traduz em maior capacidade de processamento sob estresse.

\section{Visualizações Acadêmicas Avançadas}

Para uma análise mais aprofundada dos resultados, foram geradas visualizações adicionais que complementam a análise quantitativa.

\begin{figure}[H]
\centering
\includegraphics[width=0.85\textwidth]{images/violin_latency_comparison.png}
\caption{Distribuição de Latência por Versão (Violin Plot): Comparação da densidade de distribuição dos tempos de resposta entre V1, V2 e V3 nos diferentes cenários}
\label{fig:violin_latency}
\end{figure}
\vspace{-1em}

A Figura \ref{fig:violin_latency} apresenta um \textit{violin plot} que combina box plot com estimativa de densidade kernel. Observa-se que a V2 apresenta distribuição mais concentrada em tempos baixos (corpo mais largo na parte inferior), enquanto V1 e V3 apresentam maior dispersão. A forma do ``violino'' da V2 demonstra sua consistência de performance.

\begin{figure}[H]
\centering
\includegraphics[width=0.85\textwidth]{images/heatmap_success_rate.png}
\caption{Heatmap de Taxa de Sucesso: Visualização matricial das taxas de sucesso por versão e cenário de teste}
\label{fig:heatmap_success}
\end{figure}
\vspace{-1em}

O heatmap da Figura \ref{fig:heatmap_success} permite visualizar rapidamente a performance relativa de cada versão em cada cenário. As cores mais intensas (verde/azul) indicam maior taxa de sucesso, tornando evidente o destaque da V2 em todos os cenários, especialmente os de maior estresse.

\begin{figure}[H]
\centering
\includegraphics[width=0.85\textwidth]{images/06_consolidated_metrics_radar.png}
\caption{Métricas Consolidadas (Gráfico Radar): Visão multidimensional comparando V1 e V2 em múltiplas métricas normalizadas}
\label{fig:radar_metrics}
\end{figure}
\vspace{-1em}

O gráfico radar da Figura \ref{fig:radar_metrics} oferece uma visão holística das diferenças entre V1 e V2, normalizando as métricas para permitir comparação direta. A área coberta pela V2 é consistentemente maior nas dimensões de disponibilidade, throughput e tempo de resposta, confirmando sua superioridade em múltiplas dimensões simultaneamente.

\begin{figure}[H]
\centering
\includegraphics[width=0.85\textwidth]{images/bars_success_rate.png}
\caption{Comparação de Taxas de Sucesso por Cenário (Gráfico de Barras): Análise detalhada incluindo V3 (Retry)}
\label{fig:bars_success}
\end{figure}
\vspace{-1em}

A Figura \ref{fig:bars_success} apresenta a comparação direta das taxas de sucesso entre as três versões em formato de barras agrupadas. É notável que V2 mantém 100\% em todos os cenários, enquanto V1 e V3 apresentam comportamento quase idêntico, reforçando que o padrão Retry sozinho não melhora a disponibilidade.

\section{Análise Estatística Formal}

Para validar estatisticamente a diferença entre as versões V1 e V2, aplicamos testes não-paramétricos dado o grande volume de dados (N > 380.000 requisições por versão) e a distribuição não-normal dos tempos de resposta.

\begin{table}[H]
\centering
\caption{Análise Estatística Comparativa: Tempos de Resposta V1 vs V2}
\label{tab:analise-estatistica}
\begin{tabular}{lr}
\toprule
Métrica & Valor \\
\midrule
N (V1) & 400.647 \\
N (V2) & 521.209 \\
Média (V1) & 534,34 ms \\
Média (V2) & 178,54 ms \\
Mediana (V1) & 38,16 ms \\
Mediana (V2) & 3,32 ms \\
Desvio Padrão (V1) & 997,39 ms \\
Desvio Padrão (V2) & 616,42 ms \\
P95 (V1) & 2.771,10 ms \\
P95 (V2) & 2.245,05 ms \\
P99 (V1) & 2.971,71 ms \\
P99 (V2) & 2.874,19 ms \\
\midrule
Mann-Whitney U & 413.180.104,00 \\
p-valor (Mann-Whitney) & $<$ 0,001 \\
Kolmogorov-Smirnov D & 0,5153 \\
p-valor (KS) & $<$ 0,001 \\
Cliff's Delta ($\delta$) & \textbf{0,594} \\
Interpretação Effect Size & \textbf{Grande} \\
IC 95\% Diferença & [339,74; 370,12] ms \\
\bottomrule
\end{tabular}
\end{table}

\textbf{Interpretação dos Resultados Estatísticos:}

O teste de Mann-Whitney U revela uma diferença estatisticamente significativa entre V1 e V2 (p $<$ 0,001), indicando que as distribuições de tempos de resposta são distintas. O \textbf{Cliff's Delta} ($\delta = 0,594$) classifica o \textit{effect size} como \textbf{grande}, confirmando que a diferença observada possui relevância prática substancial e não é resultado do acaso.

As melhorias observadas são expressivas:
\begin{itemize}
    \item \textbf{Tempo médio de resposta:} Redução de 534ms para 179ms (\textbf{-66,5\%})
    \item \textbf{Mediana:} Redução de 38ms para 3ms (\textbf{-91,3\%})
    \item \textbf{P95:} Redução de 2.771ms para 2.245ms (\textbf{-19,0\%})
    \item \textbf{Throughput:} Aumento de 222 req/s para 289 req/s (\textbf{+30,2\%})
\end{itemize}

A diferença média estimada entre V1 e V2 é de aproximadamente 355 ms (IC 95\%: 339,74 a 370,12 ms), o que representa uma melhoria de \textbf{66,5\%} no tempo médio de resposta. Este resultado demonstra que o Circuit Breaker não apenas protege o sistema contra falhas, mas também \textbf{melhora significativamente o desempenho} ao evitar esperas por timeouts durante períodos de indisponibilidade.

\section{Análise da Versão V3 (Retry com Backoff Exponencial)}

Para complementar a análise, foi implementada uma terceira versão do serviço de pagamento utilizando \textbf{apenas} o padrão Retry com backoff exponencial, \textbf{sem} Circuit Breaker. O objetivo foi comparar a eficácia de ambas as estratégias de resiliência.

\subsection{Configuração do Padrão Retry}
\begin{itemize}
    \item \texttt{maxAttempts}: 3 tentativas
    \item \texttt{waitDuration}: 500ms (inicial)
    \item \texttt{exponentialBackoffMultiplier}: 2 (500ms → 1s → 2s)
    \item \texttt{enableExponentialBackoff}: true
\end{itemize}

\subsection{Resultados Comparativos V3}

\begin{table}[H]
\centering
\caption{Comparação Detalhada: V3 (Retry) vs V1 (Baseline) vs V2 (Circuit Breaker)}
\label{tab:v3-comparison}
\begin{tabular}{lrrr}
\toprule
Métrica & V1 (Baseline) & V2 (CB) & V3 (Retry) \\
\midrule
Taxa de Sucesso & 89,97\% & 28,96\% & 89,99\% \\
Disponibilidade & 89,97\% & \textbf{100\%} & 89,99\% \\
Taxa de Falha & 10,03\% & \textbf{0\%} & 10,00\% \\
Tempo Médio & 534 ms & \textbf{179 ms} & 722 ms \\
Throughput & 222 req/s & \textbf{289 req/s} & 198 req/s \\
\bottomrule
\end{tabular}
\end{table}

\textbf{Observações Críticas sobre o Padrão Retry:}

\begin{enumerate}
    \item \textbf{Retry NÃO melhora disponibilidade em falhas persistentes:} A V3 apresentou taxa de sucesso idêntica à V1 (89,99\% vs 89,97\%), demonstrando que o Retry é ineficaz quando a dependência está consistentemente indisponível.
    
    \item \textbf{Retry aumenta a latência:} O tempo médio de resposta da V3 foi \textbf{35\% maior} que a V1 (722ms vs 534ms) devido às retentativas. Cada requisição que falha consome tempo adicional aguardando os retries.
    
    \item \textbf{Retry reduz throughput:} Com 198 req/s, a V3 processou \textbf{11\% menos} requisições que a V1 e \textbf{46\% menos} que a V2. O padrão Retry consome recursos (threads, conexões) com tentativas que eventualmente falham.
    
    \item \textbf{Retry pode amplificar problemas:} Com 3 tentativas por requisição, o Retry pode triplicar a carga sobre uma dependência já sobrecarregada, potencialmente agravando a situação.
\end{enumerate}

\textbf{Conclusão sobre V3:} O padrão Retry é útil para falhas \textit{transitórias} e esporádicas, mas \textbf{não substitui} o Circuit Breaker. A combinação ideal é usar Retry \textit{dentro} do Circuit Breaker, permitindo tentativas rápidas enquanto o circuito está fechado, mas acionando fallback quando a dependência está persistentemente indisponível.

\section{Discussão Geral e Impacto do Circuit Breaker}

Os resultados experimentais validam inequivocamente a eficácia do padrão Circuit Breaker e revelam as limitações do padrão Retry isolado. Os principais benefícios observados foram:

\begin{enumerate}
    \item \textbf{Prevenção de Falhas em Cascata:} O CB isolou o \texttt{servico-pagamento} das falhas do \texttt{servico-adquirente}, impedindo que a indisponibilidade se propagasse.
    \item \textbf{Degradação Graciosa:} Em vez de retornar erros HTTP 500, o sistema retornou HTTP 202 com uma mensagem útil ao usuário.
    \item \textbf{Proteção de Recursos:} O mecanismo fail-fast liberou threads rapidamente, evitando thread pool starvation.
    \item \textbf{Melhoria de Performance:} A V2 apresentou redução de 66,5\% no tempo médio de resposta e 91,3\% na mediana.
    \item \textbf{Elasticidade:} O CB demonstrou capacidade de abrir e fechar dinamicamente conforme o estado da dependência.
    \item \textbf{Aumento de Throughput:} A V2 processou 30\% mais requisições que a V1 (289 vs 222 req/s).
    \item \textbf{Superioridade sobre Retry:} Enquanto V3 (Retry) não melhorou disponibilidade, a V2 alcançou 100\% com fallback.
\end{enumerate}

\textbf{Impacto Quantificado:} A Tabela \ref{tab:impacto-quantificado} resume os principais ganhos proporcionados pelo Circuit Breaker.

\begin{table}[H]
\centering
\caption{Impacto Quantificado do Circuit Breaker}
\label{tab:impacto-quantificado}
\begin{tabular}{lrrr}
\toprule
Métrica & V1 & V2 & Melhoria \\
\midrule
Tempo Médio de Resposta & 534 ms & 179 ms & \textbf{-66,5\%} \\
Mediana de Resposta & 38 ms & 3 ms & \textbf{-91,3\%} \\
P95 & 2.771 ms & 2.245 ms & \textbf{-19,0\%} \\
Throughput & 222 req/s & 289 req/s & \textbf{+30,2\%} \\
Taxa de Falha & 10,03\% & 0\% & \textbf{-100\%} \\
Disponibilidade & 89,97\% & 100\% & \textbf{+10 pp} \\
\bottomrule
\end{tabular}
\end{table}
