\documentclass[12pt]{article}

% Essential packages (basic TeX Live)
\usepackage[utf8]{inputenc}
\usepackage[T1]{fontenc}
\usepackage[english]{babel}
\usepackage{amsmath,amssymb,amsfonts}
\usepackage{graphicx}
\usepackage{textcomp}
\usepackage{booktabs}
\usepackage{enumitem}
\setlist{nosep} % Global setting for compact lists
\usepackage{url}
\usepackage[alf]{abntex2cite}  % ABNT citations (Author-date)
\usepackage{float}
\usepackage{geometry}
\usepackage{times}
\usepackage{indentfirst}

% Page layout for UFPE TCC Article format
\geometry{
    a4paper,
    left=3cm,
    right=2cm,
    top=3cm,
    bottom=2cm
}

% Line spacing 1.5
\linespread{1.3}

% Settings
\graphicspath{{images/}}
\raggedbottom

% Custom title formatting for UFPE style
\makeatletter
\renewcommand{\maketitle}{%
    \begin{center}
        {\Large\bfseries\@title\par}
        \vskip 1.5em
        {\normalsize\@author\par}
        \vskip 2em
    \end{center}
}
\makeatother

% Section formatting
\usepackage{titlesec}
\titleformat{\section}{\normalfont\bfseries}{\thesection.}{0.5em}{}
\titleformat{\subsection}{\normalfont\bfseries}{\thesubsection}{0.5em}{}
\titleformat{\subsubsection}{\normalfont\itshape}{\thesubsubsection}{0.5em}{}

\begin{document}

\title{Quantitative Evaluation of the Circuit Breaker Pattern\\in Microservices: An Empirical Study on Resilience and Performance}

\author{
\textbf{Humberto Laff}$^{1}$, \textbf{Jamilson Ramalho Dantas}$^{1}$ (Advisor)\\[0.5em]
$^{1}$Centro de Inform\'atica (CIn)\\
Universidade Federal de Pernambuco (UFPE)\\
50740-560 -- Recife -- PE -- Brazil\\[0.5em]
\texttt{\{hlaff, jrd\}@cin.ufpe.br}
}

\maketitle

\noindent\textbf{\textit{Abstract.}} Microservices architectures have become the standard for building modern distributed systems, but synchronous communication between services introduces the risk of cascading failures. The Circuit Breaker pattern is widely recommended as a mitigation strategy, yet there is a lack of empirical studies quantifying its actual impact. This paper presents a controlled experimental evaluation comparing four service versions: V1 (Baseline without resilience), V2 (Circuit Breaker with Resilience4j), V3 (Retry with exponential backoff), and V4 (Composition of Retry + Circuit Breaker). Using Docker orchestration and k6 load testing across five realistic failure scenarios, we analyzed over 380,000 requests. Results demonstrate that Circuit Breaker improved availability from 10.3\% to 99.4\% in extreme unavailability scenarios---a gain of +89.1 percentage points. Statistical analysis confirmed a medium effect size (Cliff's Delta = 0.348, p $<$ 0.0001). Our findings provide robust empirical evidence that Circuit Breaker is essential for mission-critical synchronous microservices, while Retry alone is insufficient for persistent failures. V4 (Composition) demonstrated the highest robustness, absorbing transient jitters while maintaining protection against persistent failures.

\vspace{1em}
\noindent\textbf{Keywords:} Circuit Breaker, Microservices, Fault Tolerance, Resilience Patterns, Performance Evaluation

\vspace{2em}

% Sections
\section{Introduction}
\label{sec:introduction}

The microservices architecture has become ubiquitous in organizations building large-scale digital platforms that require continuous availability and accelerated evolution cycles \cite{newman2021}. E-commerce ecosystems and payment processing systems exemplify this movement, demanding flexibility, fault tolerance, and rapid adaptation to variable transaction volumes.

While partitioning functionality into independent services facilitates parallel development and selective scalability, this logical independence relies on real-time interactions between services, typically through REST APIs and declarative clients like Spring Cloud OpenFeign. Synchronous communication simplifies implementation and observability but introduces strong temporal coupling: the consumer service remains blocked until the dependent service responds or a timeout occurs \cite{nygard2018}.

\subsection{Problem Statement}

The risk inherent to synchronous communication constitutes the core of this investigation. When a dependent service experiences high latency or intermittent unavailability, the consuming service waits until timeout, keeping threads blocked. With increasing request volume, the thread pool exhausts (\textbf{thread pool starvation}), causing \textbf{cascading failures} that can bring down the entire system.

This problem is transversal across software engineering domains: e-commerce, logistics, healthcare, fintech, streaming, and IoT. Any distributed system relying on synchronous HTTP calls is subject to the risks analyzed here.

\subsection{Proposed Solution}

The Circuit Breaker (CB) pattern emerges as a response to these challenges. Operating as a state machine with three modes—\textbf{Closed}, \textbf{Open}, and \textbf{Half-Open}—the CB monitors calls to dependent services and interrupts new attempts when failure rates exceed configured thresholds, failing fast and protecting the consumer from resource exhaustion \cite{fowler2014cb}.

\subsection{Contribution}

Despite extensive literature on microservices and resilience patterns, there is a significant gap regarding \textbf{quantitative experimental studies} demonstrating the actual impact of the Circuit Breaker pattern. Most available documentation is limited to conceptual descriptions or trivial examples.

This paper fills this gap by:
\begin{enumerate}
    \item \textbf{Implementing} a Proof of Concept (POC) simulating a microservices ecosystem with synchronous dependency, instrumented with Resilience4j;
    \item \textbf{Executing} benchmark campaigns with Docker and k6 to \textbf{empirically measure} throughput, latency (p95), and error rates across controlled scenarios;
    \item \textbf{Comparing} three architectural versions: Baseline (V1), Circuit Breaker (V2), and Retry with exponential backoff (V3);
    \item \textbf{Providing} rigorous statistical analysis with effect size measures (Cohen's d = 1.078).
\end{enumerate}

\section{Related Work}
\label{sec:related}

The Circuit Breaker pattern was popularized by Nygard \cite{nygard2018} in ``Release It!'' and documented by Fowler \cite{fowler2014cb}. It operates as an electrical breaker: when abnormal conditions are detected, it ``opens'' to interrupt flow and protect the system.

Montesi and Weber \cite{montesi2016} analyze the interaction between Circuit Breakers and API Gateways, proposing composition patterns. Burns \cite{burns2018} contextualizes resilience patterns in modern distributed systems design.

Of particular relevance is the study by Pinheiro et al. \cite{pinheiro2024}, proposing analytical modeling of Circuit Breaker behavior using Stochastic Petri Nets (SPNs). This approach enables predicting the impact of different CB parameterizations on SLA metrics before production deployment. Our work complements this theoretical contribution by providing \textbf{empirical validation} of Circuit Breaker benefits through controlled experiments.

The taxonomy of dependability by Avizienis et al. \cite{avizienis2004} provides the theoretical framework for understanding system reliability, defining the fault-error-failure model essential for analyzing degradation in distributed systems.

Beyond library-level implementations, modern distributed systems often offload resilience logic to the \textbf{Service Mesh} layer. Istio and Linkerd provide language-agnostic circuit breaking capabilities at the infrastructure level, utilizing sidecar proxies (Envoy) to manage traffic and protect services without application code changes \cite{istio2024}. Recent research has shifted towards \textbf{adaptive circuit breaking}, where thresholds are dynamically adjusted based on real-time observability data using machine learning or PID controllers to minimize the impact of static configurations in volatile environments \cite{zhou2023adaptive}. Our study focuses on library-level implementation (Resilience4j), which remains the primary choice for granular, application-aware resilience control.

Regarding implementation, Netflix Hystrix pioneered Circuit Breaker implementation for the JVM \cite{hystrix, netflix2016}. However, in 2018, the project entered maintenance mode, and Resilience4j emerged as its recommended successor \cite{resilience4j}, offering modular design, smaller footprint, and native support for functional programming.

\section{Methodology}
\label{sec:methodology}

This work adopts a \textbf{quantitative experimental research} approach. We built a simplified Proof of Concept (POC) simulating a microservices ecosystem with synchronous dependency. The POC is intentionally minimalistic—without database, cache, or authentication—to isolate the Circuit Breaker's effect as the sole variable of interest.

\subsection{Experimental Architecture}

The POC comprises two Spring Boot microservices packaged as Docker containers:

\begin{itemize}
    \item \textbf{payment-service}: Orchestrates the payment flow and synchronously consumes the acquirer service via Feign Client;
    \item \textbf{acquirer-service}: Simulates an external payment gateway with configurable behavior (normal, latency, or failure mode).
\end{itemize}

Figure~\ref{fig:architecture} summarizes the simplified architecture used throughout the experiments.

\begin{figure}[t]
\centering
\includegraphics[width=0.48\textwidth]{simplified_architecture_en.png}
\caption{Simplified architecture of the experimental microservices ecosystem.}
\label{fig:architecture}
\end{figure}

Four versions of payment-service were developed:
\begin{itemize}
    \item \textbf{V1 (Baseline)}: Basic timeouts only (2s);
    \item \textbf{V2 (Circuit Breaker)}: Resilience4j with CB and fallback returning HTTP 202;
    \item \textbf{V3 (Retry)}: Exponential backoff retry (3 attempts, 500ms→1s→2s);
    \item \textbf{V4 (Composition)}: Combines V2 and V3, simulating a production-grade resilience stack.
\end{itemize}

\subsection{Circuit Breaker Configuration (V2)}

The Resilience4j Circuit Breaker was configured to provide a balance between fail-fast responsiveness and stability. Thresholds were selected based on the \textbf{criticality of the payment domain}.

\begin{itemize}
    \item \textbf{failureRateThreshold}: 50\% --- Represents a conservative threshold where the system assumes the dependency is no longer reliable. Lower values (e.g., 20\%) might cause \textit{flapping} due to transient jitters;
    \item \textbf{slowCallRateThreshold}: 70\% --- Prioritizes thread protection against slow dependencies, which are often more dangerous than total failures as they cause silent resource exhaustion;
    \item \textbf{slidingWindowSize}: 20 requests --- Increased from 10 to provide a more robust statistical sample, reducing the impact of outliers;
    \item \textbf{waitDurationInOpenState}: 15s --- Aligned with standard recovery times for cloud-native load balancers to detect healthy instances;
    \item \textbf{permittedNumberOfCallsInHalfOpenState}: 5 --- Allows a small "probe" traffic to verify recovery without risking a full system impact.
\end{itemize}

\subsection{Fallback Definition and RFC 9110}

The use of \textbf{HTTP 202 (Accepted)} as a fallback response is a deliberate architectural choice. According to \textbf{RFC 9110}, HTTP 202 indicates that the request has been accepted for processing, but the processing has not been completed. This aligns with a "scheduled payment" or "queue for later" pattern, maintaining semantic honesty while providing a graceful degradation path for the end user \cite{rfc9110}.

\subsection{State Transition Workflow}

The Circuit Breaker transitions through three primary states: \textbf{Closed} (all traffic flows), \textbf{Open} (traffic blocked/fallback), and \textbf{Half-Open} (controlled probing). Figure~\ref{fig:cb-workflow} illustrates this logic.

\begin{figure}[htbp]
\centering
\begin{minipage}{0.45\textwidth}
\raggedright
\small
\textbf{1. Closed} $\rightarrow$ \textbf{Open}: Triggered when $failureRate > 50\%$ in a 10-req window. \\
\textbf{2. Open} $\rightarrow$ \textbf{Half-Open}: Automatic after 10s wait duration. \\
\textbf{3. Half-Open}: Probes with 3 requests. \\
\textbf{4. Half-Open} $\rightarrow$ \textbf{Closed}: If all 3 probes succeed. \\
\textbf{5. Half-Open} $\rightarrow$ \textbf{Open}: If any probe fails.
\end{minipage}
\caption{Circuit Breaker State Machine Workflow Logic.}
\label{fig:cb-workflow}
\end{figure}

\subsection{Test Scenarios}

Five realistic failure scenarios were designed using Grafana k6 (Table~\ref{tab:scenarios}):

\begin{table}[H]
\centering
\caption{Test Scenario Characteristics}
\label{tab:scenarios}
\begin{tabular}{lccc}
\toprule
\textbf{Scenario} & \textbf{Duration} & \textbf{VUs} & \textbf{Failure Pattern} \\
\midrule
Catastrophe & 13min & 50-150 & 100\% failure for 5min \\
Degradation & 13min & 100-200 & 5\%→50\% gradual \\
Bursts & 13min & 100-200 & 3×(100\% for 1min) \\
Unavailability & 9min & 80-200 & 75\% offline \\
Normal & 10min & 100 & 100\% healthy \\
\bottomrule
\end{tabular}
\end{table}

\subsection{Metrics and Statistical Analysis}

Collected metrics include: http\_reqs (throughput), http\_req\_duration\{p(95)\} (latency percentile), and http\_req\_failed (error rate).

To reflect user-visible continuity of service, we define \textbf{Perceived Availability} as the fraction of requests that result in either a successful outcome (HTTP 200/201) or a graceful degradation outcome delivered through the fallback mechanism (HTTP 202). Following HTTP semantics, HTTP 202 (Accepted) indicates that the request has been accepted for processing, which is suitable for our “scheduled payment” fallback \cite{rfc9110}.

\[
A_p = \frac{n_{200} + n_{201} + n_{202}}{n_{\text{total}}}
\]

Statistical validation employed: Student's t-test for comparing V1 vs V2, ANOVA for three-group comparison (V1, V2, V3), and Cohen's d for effect size quantification.

\section{Results and Discussion}
\label{sec:results}

Load tests were executed using k6 and Docker Compose. Each payment-service version was submitted to all five stress scenarios. Results were evaluated against defined thresholds and analyzed in terms of success rate, response time, and fallback contribution.

\subsection{Consolidated Results Overview}

Table~\ref{tab:consolidated} presents the consolidated comparison between V1 (Baseline) and V2 (Circuit Breaker).

\begin{table}[!t]
\centering
\caption{Consolidated Comparison: V1 vs V2 by Scenario}
\label{tab:consolidated}
\small
\begin{tabular}{lcccc}
\toprule
Scenario & V1 & V2 & Fallback & Fail. Red. \\
\midrule
Catastrophe & 35.7\% & \textbf{95.1\%} & 62.1\% & -92.3\% \\
Degradation & 75.4\% & \textbf{95.4\%} & 64.7\% & -81.4\% \\
Unavail. & 10.5\% & \textbf{99.6\%} & 99.1\% & -99.5\% \\
Normal & 100.0\% & 100.0\% & 0.0\% & 0.0\% \\
Bursts & 63.0\% & \textbf{96.7\%} & 34.6\% & -91.0\% \\
\bottomrule
\end{tabular}
\end{table}

\textbf{Key Findings:}
\begin{itemize}
    \item V2 demonstrated gains of \textbf{+20pp to +89pp} in success rate across all failure scenarios;
    \item In Extreme Unavailability, V2 transformed a system with only 10.5\% success into one with 99.6\%—a \textbf{9.5x improvement};
    \item The fallback mechanism (HTTP 202) accounted for up to 99.1\% of successful responses in worst scenarios;
    \item In Normal scenario, no difference between versions confirms CB introduces \textbf{no overhead} in healthy conditions.
\end{itemize}

\subsection{V3 (Retry) Analysis}

Table~\ref{tab:v3comparison} includes V3 (Retry with exponential backoff) for comparison.

\begin{table}[!t]
\centering
\caption{Detailed Comparison: V1 vs V2 vs V3}
\label{tab:v3comparison}
\small
\begin{tabular}{lrrr}
\toprule
Scenario & V1 & V2 (CB) & V3 (Retry) \\
\midrule
Catastrophe & 35.7\% & \textbf{95.1\%} & 52.8\% \\
Degradation & 75.4\% & \textbf{95.4\%} & 76.4\% \\
Unavail. & 10.5\% & \textbf{99.6\%} & 15.7\% \\
Normal & 100.0\% & 100.0\% & 100.0\% \\
Bursts & 63.0\% & \textbf{96.7\%} & 77.7\% \\
\bottomrule
\end{tabular}
\end{table}

\vspace{1em}

\textbf{Critical Observations on Retry:}
\begin{enumerate}
    \item Retry does NOT improve availability in persistent failures;
    \item Retry increases latency due to retry attempts;
    \item Retry can amplify problems by tripling load on overloaded services.
\end{enumerate}

\subsection{Statistical Analysis}

Table~\ref{tab:statistics} presents formal statistical validation.

\begin{table}[!t]
\centering
\caption{Statistical Analysis: V1 vs V2}
\label{tab:statistics}
\small
\begin{tabular}{lr}
\toprule
Metric & Value \\
\midrule
t-test (p-value) & p $<$ 0.0001 \\
Cohen's d & 1.078 (\textbf{large}) \\
\midrule
ANOVA F-statistic & 546.79 (p $<$ 0.0001) \\
Eta-squared ($\eta^2$) & 0.267 (large) \\
\midrule
95\% CI (V1) & [495.86; 508.01] ms \\
95\% CI (V2) & [400.72; 410.62] ms \\
95\% CI (V3) & [543.38; 558.02] ms \\
\bottomrule
\end{tabular}
\end{table}

\vspace{0.5em}

The t-test reveals statistically significant difference between V1 and V2 (p < 0.0001). Cohen's d (d = 1.078) classifies the effect size as \textbf{large}, confirming substantial practical relevance. ANOVA confirmed significant difference among all three groups (F = 546.79, p < 0.0001).

\subsection{Quantified Impact}

The Circuit Breaker provided significant gains across all failure scenarios:
\begin{itemize}
    \item \textbf{Catastrophe:} +59.3pp (35.7\% $\rightarrow$ 95.1\%)
    \item \textbf{Degradation:} +20.0pp (75.4\% $\rightarrow$ 95.4\%)
    \item \textbf{Unavailability:} +89.1pp (10.5\% $\rightarrow$ 99.6\%)
    \item \textbf{Bursts:} +33.6pp (63.0\% $\rightarrow$ 96.7\%)
\end{itemize}

\subsection{Detailed Analysis: Bursts and Catastrophe}

Scenarios of \textbf{Intermittent Bursts} and \textbf{Catastrophic Failure} deserve special attention as they demonstrate the most significant Circuit Breaker benefits. Figure~\ref{fig:v1v2comparison} shows the direct comparison.

\begin{figure}[H]
\centering
\includegraphics[width=0.45\textwidth]{images/01_v1_v2_success_rate_comparison.png}
\caption{Success Rate Comparison: V1 vs V2 in Bursts and Catastrophe}
\label{fig:v1v2comparison}
\end{figure}

\textbf{Key Findings:}
\begin{itemize}
    \item In \textbf{Bursts}, V1 achieved only 63.0\% success while V2 reached \textbf{96.7\%}, a gain of \textbf{+33.6pp}.
    \item In \textbf{Catastrophe}, V1 registered 35.7\% success (system nearly unusable) while V2 maintained \textbf{95.1\%}, a gain of \textbf{+59.3pp}.
    \item Failure reduction exceeded \textbf{91\%} in both scenarios.
\end{itemize}

The Circuit Breaker achieves these results by transforming HTTP 500 errors into HTTP 202 (fallback) responses. In the Catastrophe scenario, 62.1\% of V2 responses came from fallback, effectively converting fatal errors into meaningful responses for end users.

\begin{figure}[H]
\centering
\includegraphics[width=0.48\textwidth]{images/05_combined_summary.png}
\caption{Consolidated Impact Analysis: Bursts and Catastrophe}
\label{fig:consolidated}
\end{figure}

These results validate three fundamental characteristics:
(1) \textbf{Fail-Fast with Graceful Degradation} --- the system returns meaningful alternative responses immediately;
(2) \textbf{Elasticity} --- the CB transitions dynamically between states based on dependency health;
(3) \textbf{Resource Protection} --- threads are released immediately during failures, preventing thread pool starvation.

\textbf{Sliding Window Detection Mechanism:}
A notable aspect is the CB's ability to ``anticipate'' failures through its sliding window mechanism. The CB monitors the last 10 requests and opens when failure rate exceeds 50\%. In the Catastrophe scenario, fallback rate (62.1\%) closely matched V1's actual failure rate (64.3\%), yielding a \textbf{96.6\% coverage ratio}. In Bursts, the CB demonstrated elasticity by transitioning states dynamically: fallback rate (34.6\%) matched V1 failures (37.0\%) with 93.5\% coverage, while maintaining direct success rate (62.0\%) nearly identical to V1 (63.0\%). This indicates the CB did not block requests unnecessarily during healthy periods, but correctly activated fallback during actual failures --- achieving intelligent fail-fast behavior through the half-open state mechanism that periodically probes dependency health.

\chapter{Conclusão}
\label{cap:conclusao}

\section{Revisão dos Objetivos e do Problema}
Este trabalho se propôs a investigar a fragilidade da comunicação síncrona em microsserviços, especificamente o risco de falhas em cascata em um sistema de pagamentos. O objetivo foi avaliar quantitativamente o impacto do padrão Circuit Breaker no desempenho e resiliência, usando um experimento prático e reprodutível com \texttt{Docker} e \texttt{k6}.

\section{Síntese dos Resultados}
Os resultados experimentais foram conclusivos e demonstraram de forma inequívoca o valor do padrão Circuit Breaker, bem como as limitações do padrão Retry quando usado isoladamente:

\begin{itemize}
    \item \textbf{Disponibilidade Total:} A V2 (Circuit Breaker) foi a \textbf{única} versão a alcançar 100\% de disponibilidade, eliminando completamente falhas visíveis ao usuário através do mecanismo de fallback.
    \item \textbf{Superioridade sobre Retry:} A V3 (Retry com backoff exponencial) apresentou taxa de sucesso idêntica à V1 (89,99\% vs 89,97\%), demonstrando que \textbf{Retry sozinho não melhora disponibilidade} em cenários de falha persistente.
    \item \textbf{Melhoria de Performance:} O tempo médio de resposta da V2 foi de 179ms contra 534ms da V1 (\textbf{-66,5\%}), enquanto a V3 teve o pior desempenho com 722ms (\textbf{+35\%} em relação à V1) devido às retentativas.
    \item \textbf{Aumento de Throughput:} A V2 processou 289 req/s contra 222 req/s da V1 (\textbf{+30\%}) e 198 req/s da V3 (\textbf{+46\%}).
    \item \textbf{Análise Estatística:} O teste de Mann-Whitney confirmou diferença significativa (p $<$ 0,001), e o Cliff's Delta ($\delta = 0,594$) indica \textbf{effect size grande}, confirmando relevância prática substancial e não resultado do acaso.
    \item \textbf{Elasticidade:} O CB demonstrou capacidade de transicionar dinamicamente entre estados conforme o comportamento da dependência, abrindo e fechando o circuito conforme necessário.
\end{itemize}

A arquitetura Baseline (V1) e a versão com Retry (V3) provaram ser \textbf{inadequadas para produção} em sistemas de missão crítica, enquanto a V2 (Circuit Breaker) demonstrou robustez excepcional, protegendo o \texttt{servico-pagamento} e garantindo disponibilidade percebida pelo usuário através da degradação graciosa (HTTP 202). Os resultados são particularmente relevantes para sistemas financeiros, e-commerce e qualquer domínio onde a disponibilidade é requisito crítico.

\section{Contribuições do Trabalho}
Este TCC contribui para a literatura ao fornecer:
\begin{enumerate}
    \item \textbf{Evidência Empírica Robusta:} Dados quantitativos que demonstram o impacto real do Circuit Breaker em cenários realistas de falha, com mais de 1.278.000 requisições analisadas (V1 + V2 + V3).
    \item \textbf{Comparação entre Padrões:} Primeira análise comparativa entre Circuit Breaker e Retry isolado, demonstrando que Retry \textbf{não substitui} o CB.
    \item \textbf{Análise Estatística Rigorosa:} Aplicação de testes não-paramétricos (Mann-Whitney U, Kolmogorov-Smirnov) e medidas de effect size (Cliff's Delta = 0,594 --- \textbf{grande}) que confirmam relevância prática substancial.
    \item \textbf{Metodologia Reprodutível:} Um framework de testes com Docker e k6 que pode ser replicado para avaliar outros padrões de resiliência.
    \item \textbf{Quantificação de Benefícios:} Demonstração de melhorias de 100\% de disponibilidade, 66,5\% em tempo médio de resposta e 30\% em throughput.
\end{enumerate}

\section{Limitações do Estudo}
Apesar dos resultados expressivos, este trabalho apresenta limitações que devem ser consideradas na interpretação dos achados:

\begin{enumerate}
    \item \textbf{POC Simplificada:} O sistema experimental é uma Prova de Conceito intencionalmente rudimentar, sem banco de dados, cache, autenticação ou lógica de negócio complexa. Sistemas de produção reais possuem mais variáveis que podem interagir com o comportamento do Circuit Breaker.
    
    \item \textbf{Ambiente Local:} Os experimentos foram executados em uma única máquina, sem latência de rede real entre serviços. Em ambientes distribuídos (multi-datacenter, cloud), a latência adicional pode influenciar os resultados.
    
    \item \textbf{Carga Sintética:} O k6 gera tráfego sintético com padrões uniformes. Tráfego real apresenta características mais complexas: rajadas imprevisíveis, sazonalidade e correlação entre requisições.
    
    \item \textbf{Serviço Adquirente Mockado:} O \texttt{servico-adquirente} foi implementado com falhas controladas e determinísticas. Em produção, falhas são frequentemente parciais e imprevisíveis.
    
    \item \textbf{Execução Única:} Cada cenário foi executado uma vez por versão. Múltiplas execuções permitiriam calcular intervalos de confiança e validar reprodutibilidade.
    
    \item \textbf{Configuração Fixa do CB:} Apenas uma configuração do Circuit Breaker foi testada. Diferentes parametrizações podem resultar em comportamentos distintos.
    
    \item \textbf{Domínio Específico:} Embora o contexto de pagamentos seja generalizável, outros domínios podem ter requisitos específicos (ex: latência ultra-baixa em trading) que influenciam a aplicabilidade dos resultados.
\end{enumerate}

\section{Ameaças à Validade}
Identificamos as seguintes ameaças à validade dos resultados:

\textbf{Validade Interna:}
\begin{itemize}
    \item \textbf{Variabilidade do ambiente:} Processos em segundo plano no sistema operacional, garbage collection da JVM e contenção de recursos do Docker podem introduzir variância nos resultados.
    \item \textbf{Warm-up da JVM:} A compilação JIT (Just-In-Time) pode afetar os primeiros minutos de cada teste. Mitigamos isso com fases de aquecimento nos scripts k6.
    \item \textbf{Ordem de execução:} Os testes foram executados sequencialmente; efeitos de ordem (ex: fragmentação de memória) não foram controlados.
\end{itemize}

\textbf{Validade Externa:}
\begin{itemize}
    \item \textbf{Generalização:} Os resultados são específicos para o domínio de pagamentos e a stack tecnológica utilizada (Java/Spring). Outros domínios e tecnologias podem apresentar comportamentos diferentes.
    \item \textbf{Escala:} O experimento utilizou até 200 VUs (usuários virtuais). Sistemas de produção podem enfrentar milhares de requisições simultâneas, alterando a dinâmica de contenção.
    \item \textbf{Complexidade arquitetural:} O experimento envolveu apenas dois serviços. Arquiteturas reais com dezenas de microsserviços introduzem cadeias de dependência mais complexas.
\end{itemize}

\textbf{Validade de Construção:}
\begin{itemize}
    \item \textbf{Definição de sucesso:} Consideramos HTTP 200 e HTTP 202 (fallback) como sucesso. Em contextos onde o fallback não é aceitável pelo negócio, a interpretação dos resultados seria diferente.
    \item \textbf{Métricas selecionadas:} Focamos em taxa de sucesso, latência e throughput. Outras métricas (uso de CPU, memória, conexões abertas) poderiam revelar aspectos adicionais.
\end{itemize}

\section{Trabalhos Futuros}
Como trabalho futuro, sugere-se:
\begin{enumerate}
    \item \textbf{Comparação com Outros Padrões:} Expandir o experimento para comparar o Circuit Breaker com outros padrões de resiliência como Retry (com backoff exponencial), Bulkhead (isolamento de threads) e Rate Limiter, avaliando tanto o uso isolado quanto a composição destes padrões.
    
    \item \textbf{Análise Paramétrica:} Investigar sistematicamente o impacto de diferentes configurações do Circuit Breaker (ex: \texttt{slidingWindowSize}, \texttt{failureRateThreshold}, \texttt{waitDurationInOpenState}), identificando configurações ótimas para diferentes perfis de carga.
    
    \item \textbf{Cenários de Múltiplas Dependências:} Avaliar o comportamento do CB em arquiteturas com múltiplos serviços dependentes, investigando estratégias de Circuit Breaker por dependência versus Circuit Breaker global.
    
    \item \textbf{Comunicação Assíncrona:} Comparar os resultados com arquiteturas baseadas em mensageria (Apache Kafka, RabbitMQ), avaliando os trade-offs entre comunicação síncrona protegida por CB e comunicação assíncrona nativa.
    
    \item \textbf{Ambiente Cloud Distribuído:} Replicar os experimentos em ambiente cloud (AWS, GCP ou Azure) com serviços distribuídos geograficamente, introduzindo latência de rede real e avaliando o comportamento do CB em cenários de particionamento de rede.
    
    \item \textbf{Chaos Engineering:} Integrar ferramentas de Chaos Engineering (ex: Chaos Monkey, Litmus) para injeção de falhas aleatórias e contínuas, validando a robustez do Circuit Breaker em condições mais realistas e imprevisíveis.
    
    \item \textbf{Observabilidade Avançada:} Implementar distributed tracing (Jaeger, Zipkin) para correlacionar o estado do Circuit Breaker com traces de requisições, facilitando a análise de causa raiz em cenários complexos.
    
    \item \textbf{Machine Learning para Tuning:} Explorar o uso de algoritmos de aprendizado de máquina para ajuste dinâmico dos parâmetros do Circuit Breaker com base em padrões históricos de tráfego e falhas.
    
    \item \textbf{Estudo Longitudinal:} Conduzir um estudo de longo prazo em ambiente de produção para avaliar a eficácia do Circuit Breaker ao longo de meses, capturando eventos reais de degradação e falha.
\end{enumerate}


% References
\bibliographystyle{abntex2-alf}
\bibliography{references}

\end{document}
