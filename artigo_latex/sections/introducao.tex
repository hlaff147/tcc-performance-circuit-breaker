\section{Introduction}
\label{sec:introduction}

The microservices architecture has become ubiquitous in organizations building large-scale digital platforms that require continuous availability and accelerated evolution cycles \cite{newman2021}. E-commerce ecosystems and payment processing systems exemplify this movement, demanding flexibility, fault tolerance, and rapid adaptation to variable transaction volumes.

While partitioning functionality into independent services facilitates parallel development and selective scalability, this logical independence relies on real-time interactions between services, typically through REST APIs and declarative clients like Spring Cloud OpenFeign. Synchronous communication simplifies implementation and observability but introduces strong temporal coupling: the consumer service remains blocked until the dependent service responds or a timeout occurs \cite{nygard2018}.

\subsection{Problem Statement}

The risk inherent to synchronous communication constitutes the core of this investigation. When a dependent service experiences high latency or intermittent unavailability, the consuming service waits until timeout, keeping threads blocked. With increasing request volume, the thread pool exhausts (\textbf{thread pool starvation}), causing \textbf{cascading failures} that can bring down the entire system.

This problem is transversal across software engineering domains: e-commerce, logistics, healthcare, fintech, streaming, and IoT. Any distributed system relying on synchronous HTTP calls is subject to the risks analyzed here.

\subsection{Proposed Solution}

The Circuit Breaker (CB) pattern emerges as a response to these challenges. Operating as a state machine with three modes—\textbf{Closed}, \textbf{Open}, and \textbf{Half-Open}—the CB monitors calls to dependent services and interrupts new attempts when failure rates exceed configured thresholds, failing fast and protecting the consumer from resource exhaustion \cite{fowler2014cb}.

\subsection{Contribution}

Despite extensive literature on microservices and resilience patterns, there is a significant gap regarding \textbf{quantitative experimental studies} demonstrating the actual impact of the Circuit Breaker pattern. Most available documentation is limited to conceptual descriptions or trivial examples.

This paper fills this gap by:
\begin{enumerate}
    \item \textbf{Implementing} a Proof of Concept (POC) simulating a microservices ecosystem with synchronous dependency, instrumented with Resilience4j;
    \item \textbf{Executing} benchmark campaigns with Docker and k6 to \textbf{empirically measure} throughput, latency (p95), and error rates across controlled scenarios;
    \item \textbf{Comparing} three architectural versions: Baseline (V1), Circuit Breaker (V2), and Retry with exponential backoff (V3);
    \item \textbf{Providing} rigorous statistical analysis with effect size measures (Cohen's d = 1.078).
\end{enumerate}
