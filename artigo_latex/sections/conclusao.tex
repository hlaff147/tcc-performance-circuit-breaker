\section{Conclusion}
\label{sec:conclusion}

This work investigated the fragility of synchronous communication in microservices, specifically the risk of cascading failures. The objective was to quantitatively evaluate the impact of the Circuit Breaker pattern on performance and resilience.

\subsection{Summary of Results}

Experimental results were conclusive:
\begin{itemize}
    \item \textbf{Dramatic Availability Improvement}: V2 (Circuit Breaker) achieved superior Perceived Availability in all failure scenarios. V4 (Composition) further enhanced this by absorbing transient jitters without triggering circuit state changes;
    \item \textbf{Significant Failure Reduction}: Up to 99.5\% reduction in failure rates;
    \item \textbf{Fallback as Success Strategy}: The HTTP 202 fallback mechanism accounted for up to 99.1\% of successful responses. This shifts the architectural paradigm from \textit{Strict Consistency} to \textit{Eventual Operational Continuity};
    \item \textbf{Superiority over Retry}: V3 (Retry) demonstrated a "Victim DoS" risk, where excessive retries potentially hinder the recovery of failing downstream services;
    \item \textbf{Statistical Validation}: Cohen's d = 1.078 confirms large effect size with substantial practical relevance;
    \item \textbf{Zero Overhead in Health}: No performance penalty was detected under normal operating conditions.
\end{itemize}

Beyond numbers, our study highlights the \textbf{qualitative psychological impact}: a "Fail-Fast" response (V2/V4) is systematically better for user satisfaction than a "Hang" followed by an error (V1/V3). Immediate feedback allows the application layer to offer alternatives to the user.

\subsection{Contributions}

This paper contributes to the literature by providing:
\begin{enumerate}
    \item Robust empirical evidence demonstrating real Circuit Breaker impact across five realistic failure scenarios;
    \item Comparative analysis between Circuit Breaker (V2) and isolated Retry (V3);
    \item Rigorous statistical analysis with effect size measures;
    \item Reproducible methodology with Docker and k6 that can be replicated for evaluating other resilience patterns.
\end{enumerate}

\subsection{Limitations}

This study has limitations: (i) simplified POC without database or complex business logic; (ii) local environment without real network latency; (iii) synthetic load with uniform patterns; (iv) single CB configuration tested.

\subsection{Practical Recommendations}

Based on our findings, we suggest the following best practices:
\begin{enumerate}
    \item \textbf{Combine Patterns}: Do not use Retry alone for persistent failures; wrap it \textit{inside} a Circuit Breaker to prevent resource exhaustion;
    \item \textbf{Meaningful Fallbacks}: Use HTTP 202/204 to indicate acceptance for later processing rather than generic 500 errors;
    \item \textbf{Conservative Probing}: Configure the Half-Open state with a small number of permitted calls to avoid overwhelming a recovering dependency.
\end{enumerate}

\subsection{Future Work}

Future research directions include: 
(i) evaluating \textbf{Hybrid Resilience Strategies} combining CB, Retry, and Rate Limiting; 
(ii) parametric analysis of detection windows vs. recovery time; 
(iii) integration with \textbf{Chaos Engineering} frameworks to automate failure injection; 
(iv) comparison with asynchronous architectures (Kafka, RabbitMQ) and Service Meshes (Istio, Linkerd) in multi-cloud environments.
