\section{Results and Discussion}
\label{sec:results}

Load tests were executed using k6 and Docker Compose. Each payment-service version was submitted to all five stress scenarios. Results were evaluated against defined thresholds and analyzed in terms of \textbf{Perceived Availability} (HTTP 200/201 + HTTP 202), response time, and fallback contribution.

\subsection{Consolidated Results Overview}

Table~\ref{tab:comprehensive-results} presents the consolidated performance comparison across all critical scenarios, highlighting the availability gains provided by the Circuit Breaker (V2) compared to the baseline (V1) and Retry (V3).

\begin{table*}[!t]
\centering
\caption{Comprehensive Performance Comparison: Perceived Availability across Failure Scenarios}
\label{tab:comprehensive-results}
\begin{tabular}{lccccccc}
\toprule
\textbf{Scenario} & \textbf{V1 Avail.} & \textbf{V2 Avail.} & \textbf{V2 Fallb.} & \textbf{V3 Avail.} & \textbf{V4 Avail.} & \textbf{V4 Fallb.} & \textbf{Gain (V4)} \\
\midrule
Catastrophe & 35.7\% & 95.1\% & 61.8\% & 52.8\% & 95.1\% & 63.1\% & +59.4pp \\
Degradation & 75.3\% & 95.7\% & 69.1\% & 76.8\% & 96.7\% & 74.1\% & +21.4pp \\
Unavailability & 10.3\% & 99.1\% & 96.4\% & 15.7\% & 99.4\% & 97.3\% & +89.1pp \\
Bursts & 62.9\% & 96.7\% & 34.9\% & 77.7\% & 96.6\% & 34.9\% & +33.7pp \\
Normal & 100.0\% & 100.0\% & 0.0\% & 100.0\% & 100.0\% & 0.0\% & +0.0pp \\
\bottomrule
\end{tabular}
\end{table*}


To reflect user-visible continuity of service, we define \textbf{Perceived Availability} as the fraction of requests that result in either a successful outcome (HTTP 200/201) or a graceful degradation outcome delivered through the fallback mechanism (HTTP 202). Following HTTP semantics, HTTP 202 (Accepted) indicates that the request has been accepted for processing, which is suitable for our “scheduled payment” fallback \cite{rfc9110}.

We also introduce the \textbf{Load Amplification Factor} ($L_f$), defined as the ratio between total requests sent by a resilient version versus the baseline. This metric quantifies the additional pressure exerted on downstream dependencies by retry mechanisms.

\[
A_p = \frac{n_{200} + n_{201} + n_{202}}{n_{\text{total}}}, \quad L_f = \frac{n_{\text{requests}}}{n_{\text{baseline}}}
\]

Statistical validation employed: Student's t-test for comparing V1 vs V2, ANOVA for three-group comparison (V1, V2, V3), and Cohen's d for effect size quantification.

\textbf{Key Findings (Perceived Availability):}
\begin{itemize}
    \item V2 demonstrated gains of \textbf{+20pp to +89pp} in Perceived Availability across all failure scenarios;
    \item In Extreme Unavailability, V2 transformed a system with only 10.5\% Perceived Availability into one with 99.6\%---a \textbf{9.5x improvement};
    \item The fallback mechanism (HTTP 202) accounted for up to 99.1\% of successful responses in worst scenarios;
    \item In Normal scenario, no difference between versions confirms CB introduces \textbf{no overhead} in healthy conditions.
\end{itemize}

\subsection{Statistical Analysis}

Table~\ref{tab:statistics} presents formal statistical validation of the experimental results.

\begin{table}[!t]
\centering
\caption{Statistical Analysis: V1 vs V2}
\label{tab:statistics}
\small
\begin{tabular}{lr}
\toprule
Metric & Value \\
\midrule
t-test (p-value) & p $<$ 0.0001 \\
Cohen's d & 1.078 (\textbf{large}) \\
\midrule
ANOVA F-statistic & 546.79 (p $<$ 0.0001) \\
Eta-squared ($\eta^2$) & 0.267 (large) \\
\midrule
95\% CI (V1) & [495.86; 508.01] ms \\
95\% CI (V2) & [400.72; 410.62] ms \\
95\% CI (V3) & [543.38; 558.02] ms \\
\bottomrule
\end{tabular}
\end{table}

The t-test reveals statistically significant difference between V1 and V2 (p < 0.0001). Cohen's d (d = 1.078) classifies the effect size as \textbf{large}, confirming substantial practical relevance. ANOVA confirmed significant difference among all three groups (F = 546.79, p < 0.0001).

\subsection{Advanced Analysis: States and Correlations}

To further understand the Circuit Breaker behavior, we analyzed the state transitions and metric correlations. Figure~\ref{fig:cb_states} illustrates the dynamic behavior of the CB as it transitions between states to protect the system.

\begin{figure}[htbp]
\centering
\includegraphics[width=0.48\textwidth]{images/cb_state_transitions.png}
\caption{Circuit Breaker State Transitions and Latency Profile (V2).}
\label{fig:cb_states}
\end{figure}

The "fail-fast" behavior is clearly visible when the CB enters the Open state, dropping latencies to sub-millisecond levels as requests are intercepted and converted to fallbacks. Furthermore, the correlation heatmap (Figure~\ref{fig:correlation}) confirms a strong negative correlation between Circuit Breaker activation and overall system latency during failures, validating its effectiveness in resource protection.

\begin{figure}[htbp]
\centering
\includegraphics[width=0.40 \textwidth]{images/metric_correlation_heatmap.png}
\caption{Metric Correlation Heatmap across all experimental versions.}
\label{fig:correlation}
\end{figure}

\subsection{Detailed Analysis: Bursts and Catastrophe}

Scenarios of \textbf{Intermittent Bursts} and \textbf{Catastrophic Failure} deserve special attention as they demonstrate the most significant Circuit Breaker benefits. Figure~\ref{fig:v1v2comparison} shows the direct comparison.

\begin{figure}[htbp]
\centering
\includegraphics[width=0.45\textwidth]{images/01_v1_v2_success_rate_comparison.png}
\caption{Perceived Availability Comparison: V1 vs V2 in Bursts and Catastrophe}
\label{fig:v1v2comparison}
\end{figure}

\textbf{Key Findings (Perceived Availability):}
\begin{itemize}
    \item In \textbf{Bursts}, V1 achieved only 63.0\% Perceived Availability while V2 reached \textbf{96.7\%}, a gain of \textbf{+33.6pp}.
    \item In \textbf{Catastrophe}, V1 registered 35.7\% Perceived Availability (system nearly unusable) while V2 maintained \textbf{95.1\%}, a gain of \textbf{+59.3pp}.
\end{itemize}

Failure reduction exceeded \textbf{91\%} in both scenarios. The Circuit Breaker achieves these results by transforming HTTP 500 errors into HTTP 202 (fallback) responses. In the Catastrophe scenario, 62.1\% of V2 responses came from fallback, effectively converting fatal errors into meaningful outcomes for end users.

\begin{figure}[htbp]
\centering
\includegraphics[width=0.48\textwidth]{images/05_combined_summary.png}
\caption{Consolidated Impact Analysis: Bursts and Catastrophe}
\label{fig:consolidated}
\end{figure}

Beyond the quantified metrics, the qualitative impact of \textbf{Graceful Degradation} is fundamental. By utilizing HTTP 202 (Accepted) as a fallback, the system separates \textit{Functional Availability} (direct success) from \textit{Operational Availability} (system responsiveness). While V1 users experienced 3-second timeouts leading to 500 errors, V2 users received immediate acknowledgement. This allows for asynchronous retry patterns in front-end applications, significantly improving user experience despite backend dependency failures.

\textbf{Resource Protection Analysis:}
A critical, albeit less visible, benefit is the prevention of \textbf{Thread Pool Exhaustion}. In V1, each request to a failing dependency ties up a worker thread for the full 2s timeout. With a high load, all available threads quickly become blocked, causing the entire service to hang. V2's fail-fast mechanism releases threads immediately (sub-millisecond), allowing the service to remain healthy and responsive for other, non-dependent operations.

\textbf{Recovery Time and Half-Open State:}
The Circuit Breaker's recovery is governed by the Half-Open state. Our empirical analysis indicated a recovery delta of \textbf{212.14s} in the Catastrophe scenario after the dependency resumed. This substantial delay is a consequence of the \textit{sliding window} requiring a new set of healthy samples to close the circuit, combined with the conservative probe frequency. In contrast, for the \textbf{Bursts} scenario, recovery was nearly instantaneous (0.01s) once the failure period ended, as the circuit had not yet reached the opening threshold or was already in a probing state.

\textbf{Load Amplification and Victim Denial of Service:}
A critical comparison between V2 (CB) and V3 (Retry) reveals the danger of simple retry loops. In the \textbf{Extreme Unavailability} scenario, while V2 protects the provider by failing fast, V3 maintains persistent pressure. This phenomenon, known as \textbf{Victim Denial of Service (Victim DoS)}, occurs when the retry traffic from many clients prevents a struggling service from recovering, effectively acting as a self-inflicted DDoS attack. Although our localized experiment showed a Load Amplification Factor ($L_f$) near 1.0 due to k6's request rate limiting, in real-world environments, the cumulative effect can be catastrophic.

\subsection{Composition vs. Isolation (V4)}

The introduction of \textbf{V4 (Resilience Composition)} aims to address the shortcomings of both isolation (V2) and simple persistence (V3). By combining a Circuit Breaker with a Retry mechanism (Retry wrapping the CB), the system attempts to resolve transient "blips" silently while still providing hard cut-offs for persistent failures.

\textbf{Key Insight:} V4 demonstrated the highest degree of robustness. While V2 would "trip" immediately during a momentary 1s outage, V4's retry layer absorbed the jitter without a state transition, maintaining 100% functional availability for transient issues while still activating the CB protection during the "Catastrophe" scenario.

\subsection{Architectural Recommendations Summary}

Based on the experimental data and qualitative analysis, Table~\ref{tab:recommendations} provides a summary of which strategy to employ based on the service context.

\begin{table}[H]
\centering
\caption{Architectural Recommendations for Resilience Strategies}
\label{tab:recommendations}
\small
\begin{tabular}{p{0.15\textwidth}p{0.30\textwidth}p{0.45\textwidth}}
\toprule
\textbf{Strategy} & \textbf{Best For} & \textbf{Primary Risk} \\
\midrule
V1 (Baseline) & Non-critical internal tools & Cascading failure; Thread pool starvation. \\
V2 (CB) & Highly unreliable APIs; External gateways & "Pessimistic" state changes; unnecessary fallbacks on minor jitters. \\
V3 (Retry) & Database operations; Idempotent transient blips & Load amplification; Victim DoS; Thundering Herd. \\
\textbf{V4 (Comp.)} & \textbf{Mission-critical payments; Inter-service calls} & Configuration complexity; potential for slightly higher latency on failures. \\
\bottomrule
\end{tabular}
\end{table}
