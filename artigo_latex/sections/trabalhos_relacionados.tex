\section{Related Work}
\label{sec:related}

The Circuit Breaker pattern was popularized by \citeonline{nygard2018} in ``Release It!'' and formally documented by \citeonline{fowler2014cb}. It operates analogously to an electrical breaker: when abnormal conditions are detected, it ``opens'' to interrupt flow and protect the system from further damage.

\citeonline{montesi2016} analyze the interaction between Circuit Breakers and API Gateways, proposing composition patterns that leverage both mechanisms for enhanced resilience. \citeonline{burns2018} contextualizes these resilience patterns within the broader landscape of modern distributed systems design. \citeonline{richardson2018} provides comprehensive coverage of microservices patterns including fault tolerance strategies with practical Java examples.

Of particular relevance is the work by \citeonline{pinheiro2024}, which proposes analytical modeling of Circuit Breaker behavior using Stochastic Petri Nets (SPNs). This approach enables predicting the impact of different CB parameterizations on SLA metrics prior to production deployment. Our research complements this theoretical contribution by providing \textbf{empirical validation} of Circuit Breaker benefits through controlled experiments.

The taxonomy of dependability established by \citeonline{avizienis2004} provides the theoretical framework for understanding system reliability, defining the fault-error-failure model that is essential for analyzing degradation in distributed systems. This framework distinguishes between \textit{timing failures}---where a correct response is delivered outside the acceptable time window---and \textit{content failures}, both of which are critical in payment systems operating under strict Service Level Agreements (SLAs).

The official Microsoft Azure documentation \cite{microsoftpatterns} presents the Circuit Breaker as one of the essential patterns for cloud-native applications, reinforcing its relevance in modern large-scale architectures. Recent industry reports indicate that correct implementation of this pattern can reduce error rates by up to 58\% during periods of instability \cite{ieeecloud2024}.

Beyond library-level implementations, modern distributed systems often offload resilience logic to the \textbf{Service Mesh} layer. Istio and Linkerd provide language-agnostic circuit breaking capabilities at the infrastructure level, utilizing sidecar proxies (Envoy) to manage traffic and protect services without application code changes \cite{istio2024}. Recent research has shifted towards \textbf{adaptive circuit breaking}, where thresholds are dynamically adjusted based on real-time observability data using machine learning or PID controllers, thereby minimizing the impact of static configurations in volatile environments \cite{zhou2023adaptive}. Our study focuses on library-level implementation (Resilience4j), which remains the primary choice for granular, application-aware resilience control.

Regarding implementation, Netflix Hystrix pioneered Circuit Breaker implementation for the JVM \cite{hystrix, netflix2016}. However, in 2018, the project entered maintenance mode, and Resilience4j emerged as its recommended successor \cite{resilience4j}, offering modular design, a smaller footprint, and native support for functional programming paradigms.
