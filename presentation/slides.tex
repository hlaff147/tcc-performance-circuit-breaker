% ==============================================================================
% TCC Presentation - Circuit Breaker Pattern Quantitative Evaluation
% Author: Humberto Laff
% Institution: Centro de Informática (CIn) - UFPE
% Date: January 2026
% Language: English
% ==============================================================================

\documentclass[aspectratio=169,12pt]{beamer}

% ==============================================================================
% CIn-UFPE THEME
% ==============================================================================

% Base theme
\usetheme{default}
\useinnertheme{rounded}
\useoutertheme{miniframes}

% CIn-UFPE institutional colors
\definecolor{cinred}{RGB}{206, 49, 53}
\definecolor{cinredark}{RGB}{166, 39, 43}
\definecolor{cinredlight}{RGB}{236, 89, 93}
\definecolor{ufpegray}{RGB}{64, 64, 64}
\definecolor{ufpelightgray}{RGB}{245, 245, 245}

% Auxiliary colors for charts
\definecolor{successgreen}{RGB}{46, 139, 87}
\definecolor{failred}{RGB}{180, 40, 40}
\definecolor{fallbackorange}{RGB}{230, 140, 30}
\definecolor{neutralblue}{RGB}{70, 130, 180}

% Apply colors to theme
\setbeamercolor{palette primary}{bg=cinred,fg=white}
\setbeamercolor{palette secondary}{bg=cinredark,fg=white}
\setbeamercolor{palette tertiary}{bg=ufpegray,fg=white}
\setbeamercolor{palette quaternary}{bg=cinred,fg=white}

\setbeamercolor{structure}{fg=cinred}
\setbeamercolor{frametitle}{bg=white,fg=cinred}
\setbeamercolor{title}{fg=cinred}
\setbeamercolor{subtitle}{fg=ufpegray}

\setbeamercolor{block title}{bg=cinred,fg=white}
\setbeamercolor{block body}{bg=ufpelightgray,fg=black}

\setbeamercolor{block title alerted}{bg=failred,fg=white}
\setbeamercolor{block body alerted}{bg=failred!10,fg=black}

\setbeamercolor{block title example}{bg=successgreen,fg=white}
\setbeamercolor{block body example}{bg=successgreen!10,fg=black}

\setbeamercolor{item}{fg=cinred}

% Fonts
\setbeamerfont{title}{size=\Large,series=\bfseries}
\setbeamerfont{subtitle}{size=\normalsize}
\setbeamerfont{frametitle}{size=\large,series=\bfseries}
\setbeamerfont{block title}{size=\normalsize,series=\bfseries}

% ==============================================================================
% LAYOUT CONFIGURATION
% ==============================================================================

% Remove default navigation
\setbeamertemplate{navigation symbols}{}

% Margins
\setbeamersize{text margin left=0.5cm, text margin right=0.5cm}

% Compact footer
\setbeamertemplate{footline}{
    \leavevmode%
    \hbox{%
        \begin{beamercolorbox}[wd=.33\paperwidth,ht=1.8ex,dp=0.3ex,center]{palette primary}%
            \usebeamerfont{author in head/foot}\insertshortauthor
        \end{beamercolorbox}%
        \begin{beamercolorbox}[wd=.44\paperwidth,ht=1.8ex,dp=0.3ex,center]{palette secondary}%
            \usebeamerfont{title in head/foot}\insertshorttitle
        \end{beamercolorbox}%
        \begin{beamercolorbox}[wd=.23\paperwidth,ht=1.8ex,dp=0.3ex,right]{palette tertiary}%
            \usebeamerfont{date in head/foot}\insertframenumber{} / \inserttotalframenumber\hspace*{2ex}
        \end{beamercolorbox}%
    }%
    \vskip0pt%
}

% Frame title
\setbeamertemplate{frametitle}{
    \vspace{0.5em}
    \begin{minipage}[b]{0.8\textwidth}
        {\usebeamerfont{frametitle}\insertframetitle}
    \end{minipage}
    \hfill
    \begin{minipage}[b]{0.15\textwidth}
        \raggedleft
        \includegraphics[height=0.6cm]{logo_cin.png}
    \end{minipage}
    \vspace{0.1em}
    \textcolor{cinred}{\rule{\textwidth}{1pt}}
}

% ==============================================================================
% PACKAGE
% ==============================================================================

\usepackage[utf8]{inputenc}
\usepackage[T1]{fontenc}
\usepackage{graphicx}
\usepackage{booktabs}
\usepackage{amsmath}
\usepackage{tikz}
\usepackage{pgfplots}
\pgfplotsset{compat=1.18}
\usepackage{hyperref}

% Graphics path
\graphicspath{{../artigo_latex/images/}{../slides/}{../tcc_latex/images/}}

% ==============================================================================
% DOCUMENT INFORMATION
% ==============================================================================

\title[Circuit Breaker in Microservices]{Quantitative Evaluation of the Circuit Breaker Pattern in Microservices}
\subtitle{An Empirical Study on Resilience and Performance}
\author[Humberto Laff]{Humberto Laff}
\institute[CIn-UFPE]{
    Centro de Informática (CIn)\\
    Universidade Federal de Pernambuco (UFPE)\\[1em]
    \textbf{Advisor:} Prof. Jamilson Ramalho Dantas
}
\date{January 2026}

% Logo for title graphic
\titlegraphic{
    \includegraphics[height=1.2cm]{logo_cin_ufpe.png}
}

% ==============================================================================
% BEGIN DOCUMENT
% ==============================================================================

\begin{document}

% ==============================================================================
% TITLE SLIDE
% ==============================================================================

{
\setbeamertemplate{footline}{} % Remove footer from title slide
\begin{frame}
    \begin{center}
        \vspace{-1em}
        \begin{columns}[c]
            \column{0.5\textwidth}
            \centering
            \includegraphics[height=1.2cm]{ufpe-logo.png}
            \column{0.5\textwidth}
            \centering
            \includegraphics[height=1.2cm]{logo_cin.png}
        \end{columns}
        
        \vspace{1.5em}
        
        {\usebeamerfont{title}\usebeamercolor[fg]{title}\inserttitle\par}
        
        \vspace{0.5em}
        
        {\usebeamerfont{subtitle}\usebeamercolor[fg]{subtitle}\insertsubtitle\par}
        
        \vspace{1.5em}
        
        {\usebeamerfont{author}\usebeamercolor[fg]{author}\insertauthor\par}
        
        \vspace{0.5em}
        
        {\footnotesize\usebeamercolor[fg]{institute}\insertinstitute\par}
        
        \vspace{1em}
        
        {\footnotesize\usebeamercolor[fg]{date}\insertdate\par}
    \end{center}
\end{frame}
}

% ==============================================================================
% OUTLINE
% ==============================================================================

\begin{frame}{Outline}
\tableofcontents
\end{frame}

% ==============================================================================
% SECTION 1: INTRODUCTION
% ==============================================================================

\section{Introduction}

\begin{frame}[shrink=25]{Context: Microservices Architecture}
\begin{columns}[T]
\column{0.60\textwidth}
\textbf{Why Microservices?}
\begin{itemize}
    \item Independent development \& deployment
    \item Selective scalability (scale only what's needed)
    \item Technology diversity (polyglot persistence/languages)
    \item Organizational autonomy (Two-Pizza Teams)
    \item \textit{Mention: The shift from Monoliths to Distributed Systems}
\end{itemize}

\vspace{1em}

\textbf{The Challenge:}
\begin{itemize}
    \item \alert{Synchronous communication} via REST/HTTP
    \item Strong temporal coupling between services
    \item Risk of cascading failures (The "Domino Effect")
    \item \textit{Mention: How one slow service can block the entire chain}
\end{itemize}

\column{0.38\textwidth}
\begin{block}{Key Statistics}
\small
\textbf{Downtime Cost:}\\
Banking: \alert{\$5,600-\$9,000/min}\\[0.5em]
Tech Giants: \alert{\$146k-\$450k/hour}
\end{block}

\vspace{0.5em}

\begin{exampleblock}{Research Gap}
\small
Limited \textbf{quantitative} studies on Circuit Breaker \textit{actual} impact. Most are conceptual.
\end{exampleblock}

\end{columns}
\end{frame}

\begin{frame}[shrink=15]{The Problem: Cascading Failures}
\begin{columns}[T]
\column{0.48\textwidth}
\textbf{Scenario:}
\begin{enumerate}
    \item Service A calls Service B (REST/HTTP)
    \item Service B experiences latency/failure
    \item Service A threads wait until timeout (2s-3s)
    \item Thread pool becomes exhausted (Starvation)
    \item \alert{Entire system collapses}
\end{enumerate}

\vspace{1em}

\begin{alertblock}{Thread Pool Starvation}
Blocked threads $\rightarrow$ No capacity for new requests $\rightarrow$ System-wide failure.
\textit{Mention: Even healthy endpoints stop working!}
\end{alertblock}

\column{0.48\textwidth}
\begin{figure}
\centering
\includegraphics[width=\textwidth]{arch_tcc_mermaid_en.png}
\caption{Experimental architecture: payment-service depends on acquirer-service}
\end{figure}
\end{columns}
\end{frame}

\begin{frame}[shrink=15]{The Solution: Circuit Breaker Pattern}
\begin{columns}[T]
\column{0.55\textwidth}
\textbf{Electrical Fuse Analogy:}
\begin{itemize}
    \item Detects ``failure surge'' (thresholds)
    \item ``Trips'' to cut off traffic (Fail-Fast)
    \item Prevents system-wide damage (Isolation)
    \item Periodic recovery probes (Self-healing)
\end{itemize}

\vspace{1em}

\textbf{Three States:}
\begin{itemize}
    \item \textcolor{successgreen}{\textbf{Closed}}: Normal operation (Traffic flows)
    \item \textcolor{failred}{\textbf{Open}}: Fail-fast mode (Traffic blocked)
    \item \textcolor{fallbackorange}{\textbf{Half-Open}}: Recovery probing (Testing)
\end{itemize}

\column{0.43\textwidth}
\begin{block}{Key Benefits}
\small
\begin{itemize}
    \item \textbf{Fail-Fast:} Immediate response ($<$1ms)
    \item \textbf{Resource Protection:} Threads released
    \item \textbf{Graceful Degradation:} Fallback mechanism
    \item \textbf{Auto-Recovery:} Self-healing
\end{itemize}
\end{block}

\vspace{0.5em}

\begin{exampleblock}{Implementation}
\small
Resilience4j library (Java/Spring Boot)\\
HTTP 202 (Accepted) fallback strategy
\end{exampleblock}

\end{columns}
\end{frame}

\begin{frame}[shrink=5]{Research Objectives}
\begin{block}{Primary Question}
\textbf{What is the \textit{quantitative} impact of the Circuit Breaker pattern on microservices resilience and performance?}
\end{block}

\vspace{1em}

\textbf{Specific Goals:}
\begin{enumerate}
    \item Measure availability improvement across failure scenarios
    \item Compare Circuit Breaker vs. Retry vs. Composition strategies
    \item Quantify response time reduction
    \item Provide statistical validation (effect size, significance)
    \item Demonstrate resource protection benefits
\end{enumerate}

\vspace{1em}

\begin{exampleblock}{Contribution}
\small
First comprehensive empirical study providing \textbf{robust quantitative evidence} of Circuit Breaker effectiveness in microservices
\end{exampleblock}
\end{frame}

% ==============================================================================
% SECTION 2: METHODOLOGY
% ==============================================================================

\section{Methodology}

\begin{frame}[shrink=5]{Experimental Architecture}
\begin{columns}[T]
\column{0.55\textwidth}
\textbf{Proof of Concept (POC):}
\begin{itemize}
    \item Two Spring Boot microservices
    \item Docker containerization (Reproducibility)
    \item Grafana k6 load testing (Realistic load)
    \item Simplified (no DB/cache to isolate CB effect)
\end{itemize}

\vspace{1em}

\textbf{Services:}
\begin{itemize}
    \item \textbf{payment-service}: Orchestrates payment flow
    \item \textbf{acquirer-service}: Simulates payment gateway (configurable behavior: latency, 500 errors, etc.)
\end{itemize}

\column{0.43\textwidth}
\begin{figure}
\centering
\includegraphics[width=0.95\textwidth]{arch_tcc_mermaid_en.png}
\caption{System architecture with k6 load generator}
\end{figure}
\end{columns}

\vspace{0.5em}

\begin{block}{Isolation Strategy}
\small
Minimalistic POC to isolate Circuit Breaker as the \textbf{sole variable of interest}. \textit{Mention: Why we didn't use DB/Cache.}
\end{block}
\end{frame}

\begin{frame}[shrink=20]{Four Service Versions}
\begin{table}
\centering
\small
\begin{tabular}{lll}
\toprule
\textbf{Version} & \textbf{Strategy} & \textbf{Description} \\
\midrule
\textbf{V1} & Baseline & Basic timeouts only (2s) \\
\textbf{V2} & Circuit Breaker & Resilience4j + HTTP 202 fallback \\
\textbf{V3} & Retry & Exponential backoff (3 attempts) \\
\textbf{V4} & Composition & Retry + Circuit Breaker (Layered) \\
\bottomrule
\end{tabular}
\caption{Architectural strategies compared}
\end{table}

\vspace{1em}

\begin{columns}[T]
\column{0.48\textwidth}
\begin{block}{V2 Configuration}
\small
\begin{itemize}
    \item Failure threshold: 50\%
    \item Slow call threshold: 70\%
    \item Sliding window: 20 requests
    \item Open duration: 15s
    \item \textit{Mention: Why these values were chosen.}
\end{itemize}
\end{block}

\column{0.48\textwidth}
\begin{exampleblock}{V4 Strategy}
\small
Retry layer wraps CB call\\
$\rightarrow$ Absorbs transient jitters (blips)\\
$\rightarrow$ CB protects against persistent failures
\end{exampleblock}
\end{columns}
\end{frame}

\begin{frame}[shrink=10]{Test Scenarios}
\begin{table}
\centering
\small
\begin{tabular}{lccc}
\toprule
\textbf{Scenario} & \textbf{Duration} & \textbf{VUs} & \textbf{Failure Pattern} \\
\midrule
Normal & 10min & 100 & 100\% healthy \\
Degradation & 13min & 100-200 & 5\%→50\% gradual \\
Bursts & 13min & 100-200 & 3×(100\% for 1min) \\
Catastrophe & 13min & 50-150 & 100\% failure for 5min \\
Extreme Unavailability & 9min & 80-200 & 75\% offline \\
\bottomrule
\end{tabular}
\caption{Five realistic failure scenarios}
\end{table}

\vspace{1em}

\begin{block}{Metrics Collected}
\small
\begin{itemize}
    \item \textbf{Perceived Availability:} (HTTP 200/201 + HTTP 202) / Total
    \item \textbf{Response Time:} Average \& P95 latency
    \item \textbf{Failure Rate:} HTTP 500 percentage
    \item \textbf{Fallback Contribution:} HTTP 202 percentage
\end{itemize}
\end{block}
\end{frame}

\begin{frame}[shrink=15]{Key Concepts}
\begin{columns}[T]
\column{0.48\textwidth}
\begin{block}{Perceived Availability}
\small
$$A_p = \frac{n_{200} + n_{201} + n_{202}}{n_{\text{total}}}$$

Includes both \textit{successful} and \textit{gracefully degraded} responses
\end{block}

\vspace{0.5em}

\begin{block}{Load Amplification}
\small
$$L_f = \frac{n_{\text{requests}}}{n_{\text{baseline}}}$$

Measures retry pressure on downstream services
\end{block}

\column{0.48\textwidth}
\begin{alertblock}{HTTP 202 Fallback}
\small
\textbf{RFC 9110:} ``Accepted''\\
Request accepted for \textit{later} processing\\[0.5em]
Maps to: ``Scheduled payment'' or ``Queue for retry''
\end{alertblock}

\vspace{0.5em}

\begin{exampleblock}{Total Requests}
\small
\textbf{380,000+ requests} analyzed across all scenarios
\end{exampleblock}
\end{columns}
\end{frame}

% ==============================================================================
% SECTION 3: RESULTS
% ==============================================================================

\section{Results}

\begin{frame}{Availability Comparison: All Scenarios}
\begin{figure}
\centering
\includegraphics[width=0.95\textwidth]{academic_bars_availability_en.png}
\caption{Perceived Availability (\%) by version and scenario}
\end{figure}
\end{frame}

\begin{frame}[shrink=15]{Key Results: Availability Gains}
\begin{table}
\centering
\small
\begin{tabular}{lcccc}
\toprule
\textbf{Scenario} & \textbf{V1} & \textbf{V2} & \textbf{V4} & \textbf{Gain} \\
\midrule
Normal & 100.0\% & 100.0\% & 100.0\% & --- \\
Degradation & 94.2\% & 98.8\% & 98.9\% & +4.7pp \\
Bursts & 62.9\% & 96.7\% & 96.6\% & \alert{+33.8pp} \\
Catastrophe & 35.7\% & 95.1\% & 95.1\% & \alert{+59.4pp} \\
Extreme Unavail. & 10.3\% & 99.1\% & 99.4\% & \alert{\textbf{+89.1pp}} \\
\bottomrule
\end{tabular}
\caption{Perceived Availability: V1 vs V2/V4}
\end{table}

\vspace{1em}

\begin{block}{Key Insight}
Circuit Breaker transforms a \textbf{10.3\% available} system into a \textbf{99.4\% available} system during extreme failures!
\textit{Mention: The 9.5x improvement in availability.}
\end{block}
\end{frame}

\begin{frame}{Failure Reduction}
\begin{figure}
\centering
\includegraphics[width=0.95\textwidth]{02_failure_reduction_en.png}
\caption{Failure rates and reduction percentages across scenarios}
\end{figure}
\end{frame}

\begin{frame}{Success Rate Heatmap}
\begin{figure}
\centering
\includegraphics[width=0.95\textwidth]{academic_heatmap_success_en.png}
\caption{Success rate (\%) heatmap: Scenarios × Versions}
\end{figure}
\end{frame}

\vspace{0.5em}

\begin{block}{Cliff's Delta}
\small
$d = 0.500$ \\
\alert{\textbf{Large effect size}}\\[0.5em]
Substantial practical relevance. \textit{Mention: Why p-value alone isn't enough.}
\end{block}

\column{0.48\textwidth}
\begin{block}{Response Time}
\small
\textbf{Mean (V1):} 1281.44 ms\\
\textbf{Mean (V2):} 437.67 ms\\[0.5em]
\alert{Reduction: 843.77 ms}\\[0.5em]
95\% CI: [823.42, 863.95] ms
\end{block}

\vspace{0.5em}

\begin{exampleblock}{Conclusion}
\small
Circuit Breaker provides \textbf{robust} latency reduction with high confidence.
\end{exampleblock}
\end{columns}
\end{frame}

\begin{frame}{Circuit Breaker State Transitions}
\begin{figure}
\centering
\includegraphics[width=0.95\textwidth]{cb_state_transitions.png}
\caption{Latency profiles over time (V1-V4). Red shaded areas indicate CB OPEN state.}
\end{figure}
\end{frame}

\begin{frame}{Fallback Contribution}
\begin{itemize}
    \item In \textbf{Catastrophe}: 62.1\% of responses from fallback
    \item In \textbf{Extreme Unavailability}: 99.1\% of responses from fallback
    \item Fallback mechanism is \alert{essential} for resilience
\end{itemize}
\end{frame}

\begin{frame}[shrink=20]{V3 (Retry) vs V2 (Circuit Breaker)}
\begin{columns}[T]
\column{0.48\textwidth}
\begin{alertblock}{V3 Risks}
\small
\textbf{Load Amplification:}
\begin{itemize}
    \item 3× retry attempts
    \item Persistent pressure on failing service
    \item ``Victim DoS'' effect
\end{itemize}

\vspace{0.5em}

\textbf{Inadequate for persistent failures.}
\textit{Mention: Why retrying a dead service is bad.}
\end{alertblock}

\column{0.48\textwidth}
\begin{exampleblock}{V2 Benefits}
\small
\textbf{Fail-Fast Protection:}
\begin{itemize}
    \item Immediate fallback
    \item No retry pressure
    \item Allows service recovery
\end{itemize}

\vspace{0.5em}

\textbf{Effective for persistent failures.}
\end{exampleblock}
\end{columns}

\vspace{1em}

\begin{block}{V4 (Composition): Best of Both Worlds}
\small
Retry absorbs \textit{transient jitters} $\rightarrow$ CB protects against \textit{persistent failures}.
\textit{Mention: This is the "Production-Grade" strategy.}
\end{block}
\end{frame}

\begin{frame}{V4 Composition Strategy}
\begin{figure}
\centering
\includegraphics[width=0.95\textwidth]{v4-compositions-strategy.png}
\caption{V4 architecture: Retry layer wrapping Circuit Breaker}
\end{figure}

\vspace{0.5em}

\textbf{Key Advantages:}
\begin{itemize}
    \item Absorbs momentary glitches without tripping circuit
    \item Maintains hard cutoff for persistent failures
    \item Achieved \alert{\textbf{99.4\%}} availability in Extreme Unavailability
    \item \textit{Mention: Why this is the industry standard for critical APIs.}
\end{itemize}
\end{frame}

% ==============================================================================
% SECTION 4: DISCUSSION
% ==============================================================================

\section{Discussion}

\begin{frame}{Graceful Degradation: Beyond Numbers}
\begin{columns}[T]
\column{0.55\textwidth}
\textbf{User Experience Comparison:}

\vspace{0.5em}

\alert{\textbf{V1 (Baseline):}}
\begin{itemize}
    \item Wait 2-3 seconds
    \item Receive HTTP 500 error
    \item No feedback, no alternatives
\end{itemize}

\vspace{0.5em}

\textcolor{successgreen}{\textbf{V2/V4 (Circuit Breaker):}}
\begin{itemize}
    \item Immediate response ($<$1ms)
    \item HTTP 202: ``Payment scheduled''
    \item Frontend can offer alternatives
\end{itemize}

\column{0.43\textwidth}
\begin{block}{Psychological Impact}
\small
\textbf{Fail-Fast} is systematically superior to \textbf{Hang + Error}\\[0.5em]
Users prefer \textit{immediate feedback} over long waits.
\textit{Mention: The "Perceived Performance" concept.}
\end{block}

\vspace{0.5em}

\begin{exampleblock}{Business Value}
\small
Maintains \textbf{operational continuity} during backend failures.
\end{exampleblock}
\end{columns}
\end{frame}

\begin{frame}{Resource Protection: Thread Pool}
\begin{columns}[T]
\column{0.48\textwidth}
\alert{\textbf{V1 Problem:}}
\begin{enumerate}
    \item Request to failing service
    \item Thread blocks for 2s timeout
    \item Under load: all threads blocked
    \item \textbf{Thread pool exhaustion}
    \item Entire service hangs
\end{enumerate}

\vspace{0.5em}

\begin{alertblock}{Cascading Failure}
\small
Even healthy endpoints become unavailable!
\textit{Mention: The "Blast Radius" of a failure.}
\end{alertblock}

\column{0.48\textwidth}
\textcolor{successgreen}{\textbf{V2 Solution:}}
\begin{enumerate}
    \item Circuit detects failures
    \item CB opens (fail-fast mode)
    \item Threads released $<$1ms
    \item Pool remains available
    \item Service stays healthy
\end{enumerate}

\vspace{0.5em}

\begin{exampleblock}{Protection}
\small
Service can still serve other operations independently.
\textit{Mention: Bulkhead pattern relationship.}
\end{exampleblock}
\end{columns}
\end{frame}

\begin{frame}{Architectural Recommendations}
\begin{block}{Best Practice}
\textbf{Combine patterns:} Wrap Retry \textit{inside} Circuit Breaker to prevent resource exhaustion while handling transient failures gracefully.
\end{block}
\end{frame}

\begin{frame}{Recovery Time Analysis}
\begin{columns}[T]
\column{0.48\textwidth}
\textbf{Catastrophe Scenario:}
\begin{itemize}
    \item Recovery delta: \alert{212.14s}
    \item Sliding window requires new healthy samples
    \item Conservative probe frequency
\end{itemize}

\vspace{0.5em}

\textbf{Bursts Scenario:}
\begin{itemize}
    \item Recovery delta: \textcolor{successgreen}{0.01s}
    \item Circuit not yet opened
    \item Instant recovery
\end{itemize}

\column{0.48\textwidth}
\begin{block}{Half-Open State}
\small
\textbf{Configuration:}
\begin{itemize}
    \item Wait duration: 15s
    \item Probe calls: 5
    \item All probes must succeed
\end{itemize}

\vspace{0.5em}

Balances \textit{fail-fast} protection with \textit{recovery sensitivity}
\end{block}
\end{columns}

\vspace{0.5em}

\begin{alertblock}{Trade-off}
\small
Conservative recovery $\rightarrow$ Longer downtime but prevents flapping
\end{alertblock}
\end{frame}

% ==============================================================================
% SECTION 5: CONCLUSION
% ==============================================================================

\section{Conclusion}

\begin{frame}{Summary of Contributions}
\textbf{This work provides:}
\begin{enumerate}
    \item \textbf{Robust empirical evidence} of Circuit Breaker impact across 5 realistic scenarios
    \item \textbf{Quantitative comparison:} CB vs Retry vs Composition
    \item \textbf{Statistical validation:} Large effect size (Cliff's Delta = 0.500)
\end{enumerate}

\vspace{1em}

\begin{block}{Key Finding}
Circuit Breaker improved availability from \textbf{10.3\%} to \textbf{99.4\%} in extreme failures---a gain of \alert{\textbf{+89.1 percentage points}}
\end{block}
\end{frame}

\begin{frame}[shrink=15]{Main Results Recap}
\begin{columns}[T]
\column{0.48\textwidth}
\textbf{Availability:}
\begin{itemize}
    \item Up to \alert{+89.1pp} gain
    \item Consistent across scenarios
    \item V4 highest robustness
\end{itemize}

\vspace{0.5em}

\textbf{Failure Reduction:}
\begin{itemize}
    \item Up to \alert{99.0\%} reduction
    \item HTTP 500 $\rightarrow$ HTTP 202
\end{itemize}

\column{0.48\textwidth}
\textbf{Response Time:}
\begin{itemize}
    \item \alert{843ms} average reduction
    \item High confidence (p $<$ 0.0001)
    \item Large effect size
\end{itemize}

\vspace{0.5em}

\textbf{Resource Protection:}
\begin{itemize}
    \item Thread pool preserved
    \item Fail-fast $<$1ms
    \item No overhead in health
\end{itemize}
\end{columns}
\end{frame}

\begin{frame}{Practical Recommendations}
\begin{block}{1. Combine Patterns}
Do not use Retry alone for persistent failures. Wrap it \textit{inside} a Circuit Breaker (V4 strategy).
\end{block}

\begin{block}{2. Meaningful Fallbacks}
Use HTTP 202/204 to indicate acceptance for later processing, not generic 500 errors.
\end{block}

\begin{block}{3. Conservative Probing}
Configure Half-Open state with small number of permitted calls to avoid overwhelming recovering dependency.
\end{block}
\end{frame}

\begin{frame}{Limitations}
\begin{itemize}
    \item \textbf{Synthetic load:} Uniform k6 patterns, not real user behavior
    \item \textbf{Single configuration:} Only one CB parameter set tested
\end{itemize}

\vspace{1em}

\begin{alertblock}{External Validity}
Results may vary in production environments with different characteristics, but the \textit{direction} of benefits should remain consistent.
\end{alertblock}
\end{frame}

\begin{frame}{Future Work}
\textbf{Research Directions:}
\begin{enumerate}
    \item \textbf{Hybrid Strategies:} Combining CB, Retry, and Rate Limiting
    \item \textbf{Parameter Sensitivity:} Analyzing impact of different threshold configurations
    \item \textbf{Chaos Engineering:} Automated failure injection frameworks
    \item \textbf{Service Mesh Comparison:} Istio/Linkerd vs library-level CB
    \item \textbf{Asynchronous Architectures:} Kafka/RabbitMQ resilience patterns
    \item \textbf{Adaptive Circuit Breaking:} ML-based dynamic threshold adjustment
    \item \textbf{Multi-cloud Environments:} Geographic distribution impact
\end{enumerate}
\end{frame}

\begin{frame}[shrink=5]{Final Message}
\begin{center}
\Large

\textbf{The Circuit Breaker pattern is \alert{essential} for mission-critical synchronous microservices.}

\vspace{1.5em}

\normalsize
Our empirical study provides \textbf{robust quantitative evidence} that Circuit Breaker:
\begin{itemize}
    \item Dramatically improves availability (+89pp)
    \item Reduces failures by up to 99\%
    \item Protects system resources
    \item Enables graceful degradation
\end{itemize}

\vspace{1.5em}

\textcolor{successgreen}{\textbf{Recommendation:}} Use V4 (Composition) strategy for production systems.

\end{center}
\end{frame}

% ==============================================================================
% Q&A
% ==============================================================================

\begin{frame}[plain]
\begin{center}
\Huge
\textcolor{cinred}{\textbf{Thank You!}}

\vspace{2em}

\Large
Questions?

\vspace{3em}

\normalsize
\textbf{Humberto Laff}\\
\texttt{hlaff@cin.ufpe.br}

\vspace{1em}

\textbf{Advisor:} Prof. Jamilson Ramalho Dantas\\
\texttt{jrd@cin.ufpe.br}

\vspace{2em}

\small
Centro de Informática (CIn)\\
Universidade Federal de Pernambuco (UFPE)

\end{center}
\end{frame}

% ==============================================================================
% BACKUP SLIDES
% ==============================================================================

\appendix

\begin{frame}[plain]
\begin{center}
\Huge
\textcolor{cinred}{\textbf{Backup Slides}}
\end{center}
\end{frame}

\begin{frame}{Detailed Configuration: Resilience4j}
\begin{block}{Circuit Breaker Settings}
\small
\begin{itemize}
    \item \texttt{failureRateThreshold}: 50\%
    \item \texttt{slowCallRateThreshold}: 70\%
    \item \texttt{slowCallDurationThreshold}: 1000ms
    \item \texttt{slidingWindowSize}: 20 requests
    \item \texttt{waitDurationInOpenState}: 15s
    \item \texttt{permittedNumberOfCallsInHalfOpenState}: 5
    \item \texttt{automaticTransitionFromOpenToHalfOpenEnabled}: true
\end{itemize}
\end{block}

\begin{block}{Retry Settings (V3/V4)}
\small
\begin{itemize}
    \item \texttt{maxAttempts}: 3
    \item \texttt{waitDuration}: 500ms (exponential: 500ms $\rightarrow$ 1s $\rightarrow$ 2s)
    \item \texttt{exponentialBackoffMultiplier}: 2
\end{itemize}
\end{block}
\end{frame}

\begin{frame}{Metric Correlation Matrix}
\begin{figure}
\centering
\includegraphics[width=0.85\textwidth]{metric_correlation_heatmap.png}
\caption{Correlation analysis between performance metrics}
\end{figure}

\begin{itemize}
    \item Strong negative correlation: Fallback Rate $\leftrightarrow$ Failure Rate
    \item Validates fallback mechanism effectiveness
\end{itemize}
\end{frame}

\begin{frame}{Docker Compose Architecture}
\begin{block}{Services}
\small
\begin{itemize}
    \item \textbf{payment-service-v1/v2/v3/v4}: Four versions running simultaneously
    \item \textbf{acquirer-service}: Configurable failure simulator
    \item \textbf{k6}: Load generator with custom scenarios
    \item \textbf{prometheus}: Metrics collection
    \item \textbf{grafana}: Real-time monitoring
\end{itemize}
\end{block}

\begin{exampleblock}{Reproducibility}
\small
Complete setup available: \texttt{docker-compose up}\\
All scripts automated: \texttt{./run\_all\_tests.sh}
\end{exampleblock}
\end{frame}

% ==============================================================================
% FINAL SLIDE
% ==============================================================================

{
\setbeamertemplate{footline}{}
\begin{frame}
    \begin{center}
        \vspace{2em}
        {\Huge \textcolor{cinred}{\textbf{Thank You!}}}
        
        \vspace{2em}
        
        {\large \textbf{Humberto Laff}}
        
        \vspace{0.5em}
        
        \texttt{hlaff@cin.ufpe.br}
        
        \vspace{3em}
        
        \begin{columns}[c]
            \column{0.5\textwidth}
            \centering
            \includegraphics[height=1.5cm]{ufpe-logo.png}
            \column{0.5\textwidth}
            \centering
            \includegraphics[height=1.5cm]{logo_cin.png}
        \end{columns}
        
        \vspace{1em}
        {\footnotesize Centro de Informática (CIn) — UFPE}
    \end{center}
\end{frame}
}

\end{document}
